% Options for packages loaded elsewhere
\PassOptionsToPackage{unicode}{hyperref}
\PassOptionsToPackage{hyphens}{url}
%
\documentclass[
]{book}
\usepackage{amsmath,amssymb}
\usepackage{lmodern}
\usepackage{iftex}
\ifPDFTeX
  \usepackage[T1]{fontenc}
  \usepackage[utf8]{inputenc}
  \usepackage{textcomp} % provide euro and other symbols
\else % if luatex or xetex
  \usepackage{unicode-math}
  \defaultfontfeatures{Scale=MatchLowercase}
  \defaultfontfeatures[\rmfamily]{Ligatures=TeX,Scale=1}
\fi
% Use upquote if available, for straight quotes in verbatim environments
\IfFileExists{upquote.sty}{\usepackage{upquote}}{}
\IfFileExists{microtype.sty}{% use microtype if available
  \usepackage[]{microtype}
  \UseMicrotypeSet[protrusion]{basicmath} % disable protrusion for tt fonts
}{}
\makeatletter
\@ifundefined{KOMAClassName}{% if non-KOMA class
  \IfFileExists{parskip.sty}{%
    \usepackage{parskip}
  }{% else
    \setlength{\parindent}{0pt}
    \setlength{\parskip}{6pt plus 2pt minus 1pt}}
}{% if KOMA class
  \KOMAoptions{parskip=half}}
\makeatother
\usepackage{xcolor}
\IfFileExists{xurl.sty}{\usepackage{xurl}}{} % add URL line breaks if available
\IfFileExists{bookmark.sty}{\usepackage{bookmark}}{\usepackage{hyperref}}
\hypersetup{
  pdftitle={GI Surgical Oncology},
  hidelinks,
  pdfcreator={LaTeX via pandoc}}
\urlstyle{same} % disable monospaced font for URLs
\usepackage{longtable,booktabs,array}
\usepackage{calc} % for calculating minipage widths
% Correct order of tables after \paragraph or \subparagraph
\usepackage{etoolbox}
\makeatletter
\patchcmd\longtable{\par}{\if@noskipsec\mbox{}\fi\par}{}{}
\makeatother
% Allow footnotes in longtable head/foot
\IfFileExists{footnotehyper.sty}{\usepackage{footnotehyper}}{\usepackage{footnote}}
\makesavenoteenv{longtable}
\usepackage{graphicx}
\makeatletter
\def\maxwidth{\ifdim\Gin@nat@width>\linewidth\linewidth\else\Gin@nat@width\fi}
\def\maxheight{\ifdim\Gin@nat@height>\textheight\textheight\else\Gin@nat@height\fi}
\makeatother
% Scale images if necessary, so that they will not overflow the page
% margins by default, and it is still possible to overwrite the defaults
% using explicit options in \includegraphics[width, height, ...]{}
\setkeys{Gin}{width=\maxwidth,height=\maxheight,keepaspectratio}
% Set default figure placement to htbp
\makeatletter
\def\fps@figure{htbp}
\makeatother
\setlength{\emergencystretch}{3em} % prevent overfull lines
\providecommand{\tightlist}{%
  \setlength{\itemsep}{0pt}\setlength{\parskip}{0pt}}
\setcounter{secnumdepth}{5}
\usepackage{booktabs}

\ifLuaTeX
  \usepackage{selnolig}  % disable illegal ligatures
\fi
\usepackage[]{natbib}
\bibliographystyle{apalike}

\title{GI Surgical Oncology}
\author{}
\date{\vspace{-2.5em}}

\begin{document}
\maketitle

{
\setcounter{tocdepth}{0}
\tableofcontents
}
\hypertarget{Intro}{%
\chapter{Overview}\label{Intro}}

\hypertarget{absences}{%
\section{Absences}\label{absences}}

Please notify Dr Hill, Dr Salo and Dr Squires before the beginning of the rotation if you will be away during the month. This includes vacations, meetings, interview trips, and other absences.

\hypertarget{communication}{%
\section{Communication}\label{communication}}

Please use Halo for messaging service attendings rather than SMS. Please check our status prior to messaging nights and weekends. If we are listed as unavailable, please contact another Surgical Oncology attending or the CMC ``GI MIS Blue Surgery Attending Colorectal Onc'' attending on call.

\hypertarget{inpatients}{%
\section{Inpatients}\label{inpatients}}

By in large, service attendings all wish to know about major changes in the status of our own patients. For most of the issues for which you would need to contact an attending at night, we would prefer that you HALO us directly rather than the on-call person. This desire is 24/7 (unless Halo says that attending is ``off'').

\hypertarget{er-admits}{%
\section{ER admits}\label{er-admits}}

We would ask to use a combination of communication for ER admissions. All patients must be discussed with an attending.
• If the patient is stable, does not need surgery, etc. then we would ask you to contact the attending on call for GI MIS Blue Surgery Attending Colorectal Onc. Please contact them according to that attending's preferences (page, text, call, etc).
• If the patient is unstable, may need surgery or will have ongoing and/or have intensive management needs the following day please HALO that patient's surgical oncology attending directly. The difference is that this patient is sick - we would like to know about all of our sick patients. If the Surg Onc attending is listed as ``off'' within Halo then please contact the GI MIS Blue Surgery Attending Colorectal Onc on-call attending.

\hypertarget{medical-records}{%
\section{Medical Records}\label{medical-records}}

Completing medical records in a timely fashion is critical for patient safety, billing, and compliance. Timeliness also demonstrates an understanding of how the world of surgery for which residents are being prepared functions.

\hypertarget{operative-logs}{%
\section{Operative Logs}\label{operative-logs}}

Completion of operative logs is critical for board certification of the individual resident but also has implications for the appropriate assignment of residents to surgical rotations AND impacts the ability of the residency to maintain accreditation and recruit resident candidates. Residents who find it difficult to find time to maintain operative logs may find themselves excused from the operating room to complete them. Residents are expected to complete operative logs within two weeks of the end of the rotation.

\hypertarget{case-assignment}{%
\section{Case Assignment}\label{case-assignment}}

The senior resident will be expected to make case assignments for junior residents and students. It in not necessary split the month by attending - splitting by case is acceptable as well.~ We also expect that~ both residents know all the patients rather than just for one attending. This helps with nursing questions, etc.

\hypertarget{clinic}{%
\section{Clinic}\label{clinic}}

Clinic is an important part of a surgeon's education, where decisions are made regarding diagnostic workup, patient evaluation, and treatment planning. The expectation is that all residents on the service attend clinic once per week.

\hypertarget{work-hours}{%
\section{Work Hours}\label{work-hours}}

If the service workload jeopardizes your ability to abide by the work hour restrictions, you must notify an attending so that arrangements can be made. The service attendings are committed to abiding by work hour restrictions.

\hypertarget{cmc-inpatient}{%
\chapter{CMC Inpatient}\label{cmc-inpatient}}

Colorectal Surgery (Davis/Kasten) and GI Surgical Oncology (Hill/Salo/Squires) will cover Pineville and CMC. For efficiency, the services at each hospital will merge for patient care. Each patient will continue to have an attending surgeon, but rounding and inpatient care will be provided by the service.

In general Dr Kasten will round on his own patients at CMC.

\hypertarget{admissions}{%
\section{Admissions}\label{admissions}}

Admitting Provider: ``CMC GI SURGICAL ONCOLOGY''

List Attending Surgeon in addition

Patient List is CMC GI Surgical Oncology

\hypertarget{rounds}{%
\section{Rounds}\label{rounds}}

Attending rounds for both services (CR and SurgOnc) start at 6am in STICU or 11T. Dr Salo rounds M-Tu and Drs Squires and Hill round W/Th/F alternate weeks

\hypertarget{resident-epic-teams}{%
\section{Resident Epic teams:}\label{resident-epic-teams}}

CMC LCI GI SURGICAL ONCOLOGY NEW CONSULT
CMC LCI GI SURGICAL ONCOLOGY CALL
CMC LCI GI SURGICAL ONCOLOGY TEAM
CMC AH COLORECTAL SURGERY CALL
CMC AH COLORECTAL SURGERY CALL - DAVIS/KASTEN

Residents will be assigned to Epic teams by schedule. It is critical that you notify service attendings before the start of the month to adjust the resident call schedule. Each ``shift'' is 5:50am to 6pm. At 6pm the resident Halo Teams will be forwarded to the night team.

Please append a text block to the bottom of each progress note specifying the Halo Team for that patient to facilitate communication from nursing.

\hypertarget{attending-epic-teams}{%
\section{Attending Epic teams:}\label{attending-epic-teams}}

CMC AH COLORECTAL SURGERY ATTENDING
CMC LCI GI SURGICAL ONCOLOGY ATTENDING

Monday through Thursday (24hr), please contact the attending surgeon for each patient. For Surgical Oncology, please use Haiku. For Colorectal, please use phone.

Friday and Weekend: Epic Secure Chat ``CMC GI MIS Blue Surgery Attending Colorectal Onc'' for new admissions. Please keep Surgical Oncology attending informed of inpatient issues.

\hypertarget{consults}{%
\section{Consults}\label{consults}}

Established patients and directed should be discussed with the attending surgeon.

Unassigned Colorectal: ``CMC AH COLORECTAL SURGERY ATTENDING''

Unassigned Surgical Oncology: ``CMC LCI GI Surgical Oncology Attending''

In general, benign colorectal consults are staffed by Dr Davis. Colorectal malignancies ares.
staffed as below. Esophageal and GE junction staffed by Dr Salo. Adenocarcinoma of distal stomach: Drs Salo/Squires. Gastric GIST: Drs Hill/Salo/Squires. Squires/Hill alternate week
\textbar{} \textbar{} Mon \textbar{} Tues \textbar{} Weds \textbar{} Thu \textbar{} Fri \textbar{}
\textbar---\textbar---\textbar---\textbar---\textbar---\textbar---\textbar{}
\textbar{} CR Malig \textbar{} JSH \textbar{} MHS \textbar{} MHSJSH \textbar{} JCS \textbar{} MHSJSH \textbar{}
\textbar{} GI Surg Onc \textbar{} JSH \textbar{} MHS \textbar{} MHSJSH \textbar{} JCS \textbar{} MHSJSH \textbar{}

\hypertarget{postop-clinic-appts}{%
\section{Postop Clinic Appts}\label{postop-clinic-appts}}

Postoperative patients are generally seen for a Transition of Care visit within the first week

Discharge appointments are made by sending a message in Canopy the evening prior (preferred) OR the morning of discharge before 8am to:

\begin{itemize}
\tightlist
\item
  LCI CMC GI, Clerical
\item
  Marsha Sukhdeo (Salo and Squires)
\item
  Sara Gaddy (Salo and Squires)
\item
  Rebecca Wicks (Hill)
\item
  Brenda Prieto
\item
  Brandon Galloway
\end{itemize}

Please include the following information in the Canopy Message:

\begin{itemize}
\tightlist
\item
  Name of attending
\item
  Ward from which the patient is being discharged
\item
  Desired date for appointment
\item
  Need for Wound Ostomy RN appointment at same time (essential for new stomas)
\item
  Need for bloodwork at first
\item
  Other studies to be done after discharge

  \begin{itemize}
  \tightlist
  \item
    Upper GI
  \item
    Chest X-ray
  \item
    Modified Barium Swallow
  \end{itemize}
\end{itemize}

If messages cannot be sent in time, clinic schedulers can be reached at: Marsha Sukhdeo (Salo and Squires): 980-442-6110 or (Hill): 980-442-6183.

If schedulers are not available, clinic RNs can be reached at: (Hill) 980-442-6146 or (Salo and Squires) 980-442-6143.

For patients likely to go home over the weekend or holidays, please plan to send a canopy message before 3pm on Friday or the day prior.

The scheduler will respond with a message to the discharging resident AND to the ward CNL with the appointment time, which can be included within the discharge summary. Copies of the message will also be sent to clinical nurse leaders:

\begin{itemize}
\tightlist
\item
  11Tower: Sharon Hood
\item
  6Tower: Amy Peterson
\end{itemize}

\hypertarget{conferences}{%
\section{Conferences}\label{conferences}}

\begin{itemize}
\tightlist
\item
  GI Tumor Planning Conference Monday 7-8am
\item
  Resident Teaching Conference Tuesday 7-8am 5th floor LCI II. Please review the upcoming clinic schedule and choose a case to present.
\item
  Bone and Soft Tissue Conference Friday 7-8am
\end{itemize}

\hypertarget{pineville-inpatient}{%
\chapter{Pineville Inpatient}\label{pineville-inpatient}}

\hypertarget{rounds-1}{%
\section{Rounds}\label{rounds-1}}

\begin{itemize}
\tightlist
\item
  Monday - Dr Squires
\item
  Tuesday - Dr Hill
\end{itemize}

Starting time for rounds is variable from day to day.

Maddie Georgino will help organize work and timing of rounds, etc.

\hypertarget{resident-halo-teams}{%
\section{Resident Halo teams:}\label{resident-halo-teams}}

Pine AH Colorectal Surgery 1st Call
Pine LCI GI Surgical Oncology 1st Call

Residents will be assigned to Halo teams by schedule. It is critical that you notify service attendings before the start of the month to adjust the resident Halo schedule. Each ``shift'' is 5:50am to 6pm. At 6pm the resident Halo Teams will be forwarded to the night team.

Please append a text block to the bottom of each progress note specifying the Halo Team for that patient to facilitate communication from nursing.

\hypertarget{attending-halo-teams}{%
\section{Attending Halo teams:}\label{attending-halo-teams}}

Pine AH Colorectal Surgery Attending
Pine LCI GI Surgical Oncology Attending

Monday through Thursday (24hr), please contact the attending surgeon for each patient. For Surgical Oncology, please use Halo. For Colorectal, please use phone.

Friday and Weekend: Halo ``CMC GI MIS Blue Surgery Attending Colorectal Onc'' for new admissions. Please keep Surgical Oncology attending informed of inpatient issues.

\hypertarget{consults-1}{%
\section{Consults}\label{consults-1}}

Established patients and directed should be discussed with the attending surgeon.

Unassigned Colorectal: ``Pine AH Colorectal Surgery Attending''

Unassigned Surgical Oncology: ``Pine LCI GI Surgical Oncology Attending''

\hypertarget{postop-clinic-appts-1}{%
\section{Postop Clinic Appts}\label{postop-clinic-appts-1}}

Postoperative patients are generally seen for a Transition of Care visit within the first week

Discharge appointments are made by sending a message in Canopy the evening prior (preferred) OR the morning of discharge before 8am to:

\begin{itemize}
\tightlist
\item
  Hale Mock
\item
  Syreeta McNair
\item
  Madeline Georgino
\end{itemize}

Please include the following information in the Canopy Message:

\begin{itemize}
\tightlist
\item
  Name of attending
\item
  Ward from which the patient is being discharged
\item
  Desired date for appointment
\item
  Need for Wound Ostomy RN appointment at same time (essential for new stomas)
\item
  Need for bloodwork at first visit
\item
  Other studies to be done after discharge

  \begin{itemize}
  \tightlist
  \item
    Upper GI
  \item
    Chest X-ray
  \item
    Modified Barium Swallow
  \end{itemize}
\end{itemize}

For patients likely to go home over the weekend or holidays, please plan to send a canopy message before 3pm on Friday or the day prior.

\hypertarget{conferences-1}{%
\section{Conferences}\label{conferences-1}}

\begin{itemize}
\tightlist
\item
  GI Tumor Planning Conference Monday 7-8am
\item
  Resident Teaching Conference Tuesday 7-8am 5th floor LCI II. Please review the upcoming clinic schedule and choose a case to present.
\item
  Bone and Soft Tissue Conference Friday 7-8am
\end{itemize}

\hypertarget{rounds-2}{%
\chapter{Rounds}\label{rounds-2}}

Please plan to update the attending of record about their patients in the morning.

The following format will help speed communication on rounds.

\textbf{ID}: One line description: ``Mr Glenn: PostOp day 3 after low anterior resection''

\textbf{24 hour events}: Summary of important events in prior 24 hrs

\textbf{Systems-oriented Presentation:}

Neuro: Pain control, level of alertness, psychotropic meds, sedatives, and pain meds.

CardioVascular: Vital signs (normal OR cite the range of systolic blood pressures and range of heart rate). Heart rhythm. Cardiac meds. Most recent recommendations of cardiology consult.

Respiratory: Pulmonary exam, oxygen saturation, supplied oxygen, ventilator setting. Results of CXR.

GI: Diet, bowel function, NG output. Drain outputs can often be summarized unless they are unusually high or low (and ready to be removed. New finding of bile in any abdominal drain needs special emphasis. GI meds (eg protonix, Entereg). Tube feed formula, rate and duration (continuous or nocturnal). Status of C Diff tests. Results of JP drain amylase levels (gastroesophageal patients). Results of JP triglycerides or creatinine, if sent,

Renal: Urine output in 24 hours AND in most recent 8 hour shift. Presence (or absence) of Foley catheter and plans for removal, if present. Most recent creatinine. If diuretics administered, dosage and amount of urine output during the shift when it was administered. Most recent potassium in any patient receiving (or about to receive) furosemide (Lasix). Results of Mg and Phos if abnormal.

Heme: Hemoglobin, platelets, DVT prophylaxis. PLEASE CHECK THE MAR SUMMARY DAILY to be certain that the ordered DVT prophylaxis has been given.

ID: WBC, Tmax in past 24 hours, culture results.

Endo: Diabetic regimen, blood sugar range, and amount of sliding scale insulin administered in the prior 24 hours.

\textbf{Problem-Oriented Plan:}

Each of the patients problems are addressed with an assessment and plan. Pre-existing medical problems and postoperative complications need to be addressed in the plan

\hypertarget{progress-notes}{%
\chapter{Progress Notes}\label{progress-notes}}

Progress notes should include a comprehensive summary of pertinent information for the patient's medical care.

A running summary of events is recorded in the Shared Hospital Course under Inpatient Workflow

Each day, this section is carried from note to note so that each progress note contains the cancer history and the daily events.

The Progress Note then includes the date and a one-line summary of the operation performed and any intraoperative complications or events which impact the post-operative care.

Each day, an additional line is added to the Shared Hospital Course which summarizes events for that day. This makes it possible to see within each Progress Note the pertinent events for the hospitalization. These events would include extubation, re-intubation, positive cultures, dates lines are inserted or removed, dates of removal of NG tubes and drains, and transfer to ward or re-admission to ICU. This chronology assists in treatment decisions (``how old is the IJ line'' or ``when did we start antibiotics?'' or ``when is the planned antibiotic stop date''?) but also makes the discharge summary much easier to prepare.

Assessment/Plan
The medical problems currently being managed are addressed in the assessment.

\textbf{Esophagectomy Events to be Documented (in Shared Hospital Course):}

\begin{itemize}
\tightlist
\item
  Extubation date/time
\item
  NG Removal date
\item
  Chest tube removal date
\item
  MBS date(s) and results (aspiration \textbar{} penetration)
\item
  ICU DC orders written
\item
  ICU discharge (transfer to ward)
\end{itemize}

\textbf{Esophagectomy Complications to be Documented (in Shared Hospital Course):}

\begin{longtable}[]{@{}ll@{}}
\toprule
\endhead
N & Delirium \\
& Stroke \\
CV & New arrhythmia req Rx \\
& MI \\
R & Pneumonia (3 of fever \textbar{} WBC \textbar{} infiltrate \textbar{} abx \textbar{} sputum cx) \\
& Effusion req drainage \\
& Reintubation \\
& Atelectasis req bronchoscopy \\
& ARDS \\
& PE \\
& Ventilation \textgreater48 hours after leaving OR \\
GI & Anastomotic leak (medical rx \textbar{} stent \textbar{} surgery) \\
& Delayed gastric emptying req botox or NG \textgreater7d \\
& C Diff \\
GU & Urinary Retention \\
& Discharge with foley catheter \\
H & DVT req treatment \\
& Return to OR \\
& Return to ICU \\
\bottomrule
\end{longtable}

\textbf{Communication}

Please add an addendum at the BOTTOM of each progress note which includes a means for contacting the team:

Please message ``CMC LCI GI Surgical 1st Call'' via Halo 24/7. Messages are automatically forwarded to the General Surgery Resident on Call evenings and weekends.

\#Signout

\textbf{Evening Signout}

The Handoff Tool should be completed for all inpatients, and the responsible attending designated. This tool is critical for the safe care of patients by the nigh team. If there are studies which are pending at the time of signout (CT scan, follow-up Hb), it is critical that a plan be in place for whom to notify with an abnormal or critical study. In general, Drs Hill and Salo are always available until 10pm. Attending notification plans (service attending vs covering attending) for unstable patients should be negotiated before nightfall.

\textbf{Weekend Signout}

The chief resident is responsible for making certain that the weekend rounding resident is familiar with the patients, their problems, and the plan of care. A signout email should be prepared Friday afternoon and forwarded to the service attendings by 6pm for their review. This signout can then be edited with the attendings' notes and forwarded to the weekend rounding attending.

\hypertarget{discharges}{%
\chapter{Discharges}\label{discharges}}

\textbf{Discharge Prescriptions}

Prescriptions should be ideally be prepared the day prior to anticipated discharge and sent to the patient's pharmacy. According to \href{https://www.ncmedboard.org/images/uploads/license_applications/STOPAct-focusedFAQs-Sept2019.pdf}{North Carolina STOP guidelines}, opioid prescriptions for postoperative patients should be for no more than a 7 day supply.

\textbf{Additional Appointments}

If followup appointments in additional to surgical followup are needed, these should be designated on the discharge orders. Particularly:

\begin{itemize}
\tightlist
\item
  Primary Care Physician
\item
  Cardiologist (if new cardiac medicines)
\item
  Co-surgeons (Urology, Thoracic Surgery, GYN)
\end{itemize}

\textbf{Discharge Summary}

The discharge summary documents important events and complications in the postoperative course and serves to inform the referring physician and primary physician about these events, but also serves as a blueprint for post-discharge treatment planning. Please recognize that the first post-operative visit may be with a resident who may be meeting the patient for the first time. Key items to include:

\begin{longtable}[]{@{}
  >{\raggedright\arraybackslash}p{(\columnwidth - 2\tabcolsep) * \real{0.0580}}
  >{\raggedright\arraybackslash}p{(\columnwidth - 2\tabcolsep) * \real{0.9420}}@{}}
\toprule
\endhead
N & Followup plan for chronic pain management \\
& Stroke \\
CV & Postop Arrhythmia? \textbar{} MI? \textbar{} CHF? \\
& If new cardiac meds: Who is managing medications \\
& If afib: CHADS score and anticoagulation plan \\
R & Pneumonia? \textbar{} ARDS? \textbar{} TRACH? \\
& Need for home oxygen? \\
& CXR needed at first postop visit? \\
GI & Delayed gastric emptying? \textbar{} leak? \textbar{} ileus? \\
& Tube feed regimen \\
& Diet at discharge (Low residue \textbar{} Full liquds \textbar{} Meds with thickened water \textbar NPO) \\
& New stoma (ileostomy \textbar{} colostomy) \\
& Wound care needs (VAc \textbar{} Prevena) \\
GU & Urinary Retention \\
& Discharge with foley catheter \\
H & Complications: DVT \textbar{} PE \\
& Anticoagulation Plan \\
Endo & Insulin regimen at DC (dose will be in med rec) \\
ID & Antibiotics at DC \\
& Return to ICU \\
\bottomrule
\end{longtable}

\textbf{Communication}

It is essential that discharge summaries be sent to the patient's primary MD and referring physician. Please review the initial consultation note for the names of providers involved in a patient's care.

\hypertarget{education}{%
\chapter{Education}\label{education}}

\hypertarget{student-emr-notes-policy}{%
\subsection{Student EMR Notes Policy}\label{student-emr-notes-policy}}

Approved: Medical Education Division of Canopy Team 11/17/13

Students:

Should write Notes in the EMR as a part of their training. Students must sign and forward their notes. It is imperative that student notes not be saved and forwarded prior to signing the note or the authorship of the note will change. Signed notes should be forwarded to the supervising physician as soon as possible.

Supervising physicians:

Attending must sign the necessary documents per the medical staff bylaws (e.g., H \& P's and Discharge Summaries). Attendings, residents, or fellows can cosign progress notes. Each attending should clarify who is responsible for signing progress notes.

Supervising physicians have 2 options:

Correct the note as needed, append an attestation, and sign within 24 hours. When appending the student's note, the supervising physician may utilize only the student's past medical history, family history, social history, and review of systems. This is the preferred model and the most efficient. Please see example below.*

Correct the note as needed, attest with reference to supervising physician's own note, and sign within 24 hours.

Example Attending Attestation*

I have discussed the presentation of the patient with the student. I have also independently seen and examined the patient. Please see my documentation below:

History: {[}As obtained by the provider{]}

Past History, Family History, Social History, and ROS: Please see the student's note above.

Physical Exam:

General: Well-developed, well-nourished

HEENT: Pupils equal react to light extra ocular motion intact, conjunctiva pink, anicteric sclera, oropharynx clear, moist mucosa

Neck: Supple, no meningeal signs

Lungs: Clear to auscultation bilaterally, bilateral breath sounds, normal effort

Cardiac: Regular rate normal rhythm, no murmur gallop or rub

Abdomen: Soft, benign, non-distended, non-tender, no mass

Extremities: Moves all 4 extremities, pulses 2+ and equal

Neurologic: Awake, alert, and oriented x3, cranial nerves II through XII are intact, moves all 4 extremities equally, speech is normal, affect pleasant

Skin: No rash

Labs, X-rays, Tests: {[}Pertinent comments regarding positives/negatives and interpretations{]}

Medical Decision Making: {[}MDM also should be attending documentation{]}

Electronically Signed: Dr.~Supervision

\hypertarget{mie_postop}{%
\chapter{Postop Care after Esophagectomy}\label{mie_postop}}

\hypertarget{weaning-tube-feeds-in-diabetics}{%
\section{Weaning Tube Feeds in Diabetics}\label{weaning-tube-feeds-in-diabetics}}

As outpatients begin eating more orally, their tube feeds are reduced.

Weaning from 5 cans to 4 cans: Easiest method is to maintain the same schedule (16 hours) and reduce insulin dosage by 20\%. For instance, the above patient who is on 5 cans at 75mL/hour x 16 hours is receiving 16N + 8R at start of tube feeds and 6 hours later. This patient could be weaned by reducing rate form 75mL/hour x 16 hours to 60mL/hour x 16 hours and reducing insulin to 12N + 6R at the start of tube feeds AND another dose of 12N + 6R after 6 hours.

Weaning from 4 cans to 3 cans: One option is to decrease the duration of tube feeds from 16 hours to 12 hours, while maintaining rate of 60mL/hour. In this case, the insulin dosage could be kept the same at the start of tube feeds, BUT the dose 6 hours after the start of tube feeds could be omitted.

Once patients are on 3 cans per night, further weaning can be accomplished by skipping tube feeds (and insulin) every other night in a ``tube feed holiday''. This allows an evening of interrupted sleep and can tend to increase the appetite the morning after tube feeds are held.

\hypertarget{part-postoperative-care}{%
\part*{Postoperative Care}\label{part-postoperative-care}}
\addcontentsline{toc}{part}{Postoperative Care}

\hypertarget{colectomy}{%
\chapter{Colectomy}\label{colectomy}}

\textbf{Clinic}

\begin{itemize}
\tightlist
\item
  Opiod Cessation
\item
  Smoking Cessation
\item
  EtOH/Drugs of Abuse -- Social Work
\item
  Nutritonal Evaluation

  \begin{itemize}
  \tightlist
  \item
    All Patients -- Ensure for 3d preop
  \item
    Poor nutrition -- ?Delay surgery
  \end{itemize}
\item
  Preop Anesthesia/ERAS Class
\item
  Expectations of Surgery:

  \begin{itemize}
  \tightlist
  \item
    Length of Stay 1-3 days
  \item
    Diet (self-limiting)
  \item
    Pain control (low-opoid)
  \item
    Activity (OOB at 6am, OOB 3x/day)
  \end{itemize}
\item
  Bowel Prep -- Abx and mechanical
\end{itemize}

\textbf{Preop Holding}

\begin{itemize}
\tightlist
\item
  Colon PowerPlan (Hill)
\item
  Antibiotic PowerPlan
\item
  No PCN only for severe allergy
\item
  Entereg (if no preop opiods)
\item
  VTE prophylaxis
\item
  Carbohydrate load (2hr preop)
\end{itemize}

\textbf{OR}

\begin{itemize}
\tightlist
\item
  Goal-directed fluid administration
\item
  2L total in OR
\item
  3L total/first 24hrs
\item
  Open procedures: No epidural
\end{itemize}

\textbf{Postop Day 0}

\begin{itemize}
\tightlist
\item
  Goal-directed fluid administration
\item
  OOB (Dangling not compliant)
\item
  Diet

  \begin{itemize}
  \tightlist
  \item
    Low Residue diet
  \item
    Ensure Supplements
  \end{itemize}
\item
  Teaching
\item
  Gum, Mag \& Entereg
\item
  Pain Management

  \begin{itemize}
  \tightlist
  \item
    PCA
  \item
    Tylenol 1gm q6 ATC
  \item
    Gabapentin 300mg TID
  \item
    Tramadol PRN
  \item
    Resume all baseline pain meds
  \end{itemize}
\item
  Home Medications

  \begin{itemize}
  \tightlist
  \item
    Resume all home medications
  \item
    No therapeutic anti-coagiation
  \item
    Diabetes medicines

    \begin{itemize}
    \tightlist
    \item
      Prefer to resume all oral diabetes meds
    \item
      Sliding-scale insulin ordered for all diabetics
    \end{itemize}
  \end{itemize}
\end{itemize}

\textbf{Postop Day 1}

\begin{itemize}
\tightlist
\item
  Labs: K+ and CBC only
\item
  d/c ``pre'' plan \& colon visit
\item
  Heparin lock IV
\item
  Remove Foley
\item
  d/c PCA (unless laparotomy)
\item
  Out of bed \textgreater{} 6 hrs
\item
  Diet: low residue
\item
  Pain management

  \begin{enumerate}
  \def\labelenumi{\arabic{enumi})}
  \tightlist
  \item
    d/c PCA
  \item
    Tylenol
  \item
    Gabapentin 300 tid
  \item
    Tramadol
  \item
    Home meds
  \item
    Oxy if lots of pain
  \end{enumerate}
\item
  Discharge planning
\end{itemize}

\begin{enumerate}
\def\labelenumi{\arabic{enumi})}
\tightlist
\item
  Possible home today!(25\% will go home POD1)
\item
  Check patient 2-3PM
\end{enumerate}

\begin{itemize}
\tightlist
\item
  Afternoon rounds
\end{itemize}

\begin{enumerate}
\def\labelenumi{\arabic{enumi})}
\tightlist
\item
  Possible home today!
\item
  Check patient 2-4PM
\item
  Ambulation 2x's by PM
\item
  Patient education
\end{enumerate}

\textbf{Postop Day 2}

\begin{itemize}
\tightlist
\item
  No IV fluids unless indicated
\item
  No labs unless indicated
\item
  OOB \textgreater6 hrs
\item
  No PT unless going to rehab or SNF (ask Hill first)
\item
  Pain management

  \begin{itemize}
  \tightlist
  \item
    Same as POD\#1
  \end{itemize}
\item
  Discharge planning

  \begin{itemize}
  \tightlist
  \item
    Consider early D/C (median LOS=2d)
  \end{itemize}
\item
  Otherwise plan

  \begin{itemize}
  \tightlist
  \item
    Patient
  \item
    Nursing
  \end{itemize}
\item
  Afternoon rounds
\end{itemize}

\begin{enumerate}
\def\labelenumi{\arabic{enumi})}
\tightlist
\item
  Possible home today!
\item
  Check patient 2-3PM
\item
  Ambulation 2x's by PM
\item
  Check for dehydration
\item
  Patient education
\item
  Give estimated date of discharge
\end{enumerate}

\textbf{Postop Day 3-5}

\begin{itemize}
\item
  IVF

  \begin{itemize}
  \tightlist
  \item
    Consider bolus of IV fluids
  \item
    Consider maintenance IV if we feel won't resolve soon
  \end{itemize}
\item
  Labs

  \begin{itemize}
  \tightlist
  \item
    No labs unless indicated
  \item
    Consider ordering QOD Chem7 if prolonged ileus
  \item
    OOB \textgreater6 hrs
  \item
    No PT unless likely to go to rehab or SNIF (ask Hill first)
  \end{itemize}
\item
  Pain management

  \begin{itemize}
  \tightlist
  \item
    Same as POD\#1
  \end{itemize}
\item
  Afternoon rounds

  \begin{enumerate}
  \def\labelenumi{\arabic{enumi})}
  \tightlist
  \item
    Possible home today!
  \item
    Check patient 2-3PM
  \item
    Ambulation 2x's by PM
  \item
    Check for dehydration
  \item
    Patient education
  \item
    Give estimated date of discharge
  \end{enumerate}
\end{itemize}

\hypertarget{lar-ileostomy}{%
\chapter{LAR + Ileostomy}\label{lar-ileostomy}}

\textbf{Clinic}

\begin{itemize}
\tightlist
\item
  Opiod Cessation
\item
  Smoking Cessation
\item
  EtOH/Drugs of Abuse -- Social Work
\item
  Nutritional Evaluation

  \begin{itemize}
  \tightlist
  \item
    All Patients -- Ensure for 3d preop
  \item
    Poor nutrition -- ?Delay surgery
  \end{itemize}
\item
  \emph{Arrange Wound Ostomy Nursing}
\item
  \emph{Pre-approval for Home Health}
\item
  Preop Anesthesia/ERAS Class
\item
  Expectations of Surgery:

  \begin{itemize}
  \tightlist
  \item
    Length of Stay 1-3 days
  \item
    Diet (self-limiting)
  \item
    Pain control (low-opoid)
  \item
    Activity (OOB at 6am, OOB 3x/day)
  \end{itemize}
\item
  Bowel Prep -- Abx and mechanical
\end{itemize}

\textbf{Preop Holding}

\begin{itemize}
\tightlist
\item
  Colon PowerPlan (Hill)
\item
  \emph{CCM order}
\item
  \emph{WOCN order}
\item
  Antibiotic PowerPlan
\item
  No PCN only for severe allergy
\item
  Entereg (if no preop opiods)
\item
  VTE prophylaxis
\item
  Carbohydrate load (2hr preop)
\item
  Confirm stoma marking
\item
  Confirm CCM aware of ostomy
\end{itemize}

\textbf{OR}

\begin{itemize}
\tightlist
\item
  Goal-directed fluid administration

  \begin{itemize}
  \tightlist
  \item
    2L total in OR
  \item
    3L total/first 24hrs
  \end{itemize}
\item
  Open procedures: No epidural
\end{itemize}

\textbf{Postop Day 0}

\begin{itemize}
\tightlist
\item
  Goal-directed fluid administration
\item
  OOB (Dangling not compliant)
\item
  Diet

  \begin{itemize}
  \tightlist
  \item
    Low Residue diet
  \item
    Ensure Supplements
  \end{itemize}
\item
  Teaching
\item
  Gum, Mag \& Entereg
\item
  Pain Management

  \begin{itemize}
  \tightlist
  \item
    PCA
  \item
    Tylenol 1gm q6 ATC
  \item
    Gabapentin 300mg TID
  \item
    Oxycodone once at night
  \item
    Tramadol PRN
  \item
    Resume all baseline pain meds
  \end{itemize}
\item
  Home Medications

  \begin{itemize}
  \tightlist
  \item
    Resume all home medications
  \item
    No therapeutic anti-coagiation
  \item
    Diabetes medicines

    \begin{itemize}
    \tightlist
    \item
      Prefer to resume all
    \item
      Sliding-scale insulin if needed
    \end{itemize}
  \end{itemize}
\item
  \emph{Wound Ostomy teaching}
\item
  \emph{CCM for Home Health for stoma care}
\end{itemize}

\textbf{Postop Day 1}

\begin{itemize}
\tightlist
\item
  Labs: K+ and CBC only
\item
  d/c ``pre'' plan \& colon visit
\item
  Heparin lock IV
\item
  Remove Foley
\item
  d/c PCA (unless laparotomy)
\item
  Out of bed \textgreater{} 6 hrs
\item
  Diet: low residue
\item
  Pain management

  \begin{enumerate}
  \def\labelenumi{\arabic{enumi})}
  \tightlist
  \item
    d/c PCA
  \item
    Tylenol
  \item
    Gabapentin 300 tid
  \item
    Tramadol
  \item
    Home meds
  \item
    Oxy if lots of pain
  \end{enumerate}
\item
  Discharge planning
\end{itemize}

\begin{enumerate}
\def\labelenumi{\arabic{enumi})}
\tightlist
\item
  Possible home today!(25\% will go home POD1)
\item
  Check patient 2-3PM
\end{enumerate}

\begin{itemize}
\tightlist
\item
  Afternoon rounds
\end{itemize}

\begin{enumerate}
\def\labelenumi{\arabic{enumi})}
\tightlist
\item
  Possible home today!
\item
  Check patient 2-4PM
\item
  Ambulation 2x's by PM
\item
  Patient education
\end{enumerate}

\textbf{Postop Day 2}

\begin{itemize}
\tightlist
\item
  No IV fluids unless indicated
\item
  No labs unless indicated
\item
  OOB \textgreater6 hrs
\item
  No PT unless going to rehab or SNF (ask Hill first)
\item
  Pain management

  \begin{itemize}
  \tightlist
  \item
    Same as POD\#1
  \end{itemize}
\item
  \emph{Wound Ostomy Teaching}
\item
  Discharge planning

  \begin{itemize}
  \tightlist
  \item
    Consider early D/C (median LOS=2d)
  \end{itemize}
\item
  Otherwise plan

  \begin{itemize}
  \tightlist
  \item
    Patient
  \item
    Nursing
  \item
    \emph{CCM for home health for stoma care}
  \end{itemize}
\item
  Afternoon rounds
\end{itemize}

\begin{enumerate}
\def\labelenumi{\arabic{enumi})}
\tightlist
\item
  Possible home today!
\item
  Check patient 2-3PM
\item
  Ambulation 2x's by PM
\item
  Check for dehydration
\item
  Patient education
\item
  Give estimated date of discharge
\end{enumerate}

\textbf{Postop Day 3-5}
- IVF
- Consider bolus of IV fluids
- Consider maintenance IV if we feel won't resolve soon
- Labs
- No labs unless indicated
- Consider ordering QOD Chem7 if prolonged ileus
- OOB \textgreater6 hrs
- No PT unless likely to go to rehab or SNIF (ask Hill first)
- Pain management
- Same as POD\#1
- Afternoon rounds
1) Possible home today!
2) Check patient 2-3PM
3) Ambulation 2x's by PM
4) Check for dehydration
5) Patient education
6) Give estimated date of discharge

\hypertarget{abdominoperineal-resection}{%
\chapter{Abdominoperineal Resection}\label{abdominoperineal-resection}}

\textbf{Clinic}

\begin{itemize}
\tightlist
\item
  Opiod Cessation
\item
  Smoking Cessation
\item
  EtOH/Drugs of Abuse -- Social Work
\item
  Nutritional Evaluation

  \begin{itemize}
  \tightlist
  \item
    All Patients -- Ensure for 3d preop
  \item
    Poor nutrition -- ?Delay surgery
  \end{itemize}
\item
  \emph{Arrange Wound Ostomy Nursing}
\item
  \emph{Pre-approval for Home Health}
\item
  Preop Anesthesia/ERAS Class
\item
  Expectations of Surgery:

  \begin{itemize}
  \tightlist
  \item
    Length of Stay 1-3 days
  \item
    Diet (self-limiting)
  \item
    Pain control (low-opoid)
  \item
    Activity (OOB at 6am, OOB 3x/day)
  \end{itemize}
\item
  Bowel Prep -- Abx and mechanical
\end{itemize}

\textbf{Preop Holding}

\begin{itemize}
\tightlist
\item
  Colon PowerPlan (Hill)
\item
  \emph{CCM order}
\item
  \emph{WOCN order}
\item
  Antibiotic PowerPlan
\item
  No PCN only for severe allergy
\item
  Entereg (if no preop opiods)
\item
  VTE prophylaxis
\item
  Carbohydrate load (2hr preop)
\item
  Confirm stoma marking
\item
  Confirm CCM aware of ostomy
\end{itemize}

\textbf{OR}

\begin{itemize}
\tightlist
\item
  Goal-directed fluid administration

  \begin{itemize}
  \tightlist
  \item
    2L total in OR
  \item
    3L total/first 24hrs
  \end{itemize}
\item
  Open procedures: No epidural
\end{itemize}

\textbf{Postop Day 0}

\begin{itemize}
\tightlist
\item
  No sitting

  \begin{itemize}
  \tightlist
  \item
    Order Sign over bed ``No sitting''
  \end{itemize}
\item
  Goal-directed fluid administration
\item
  OOB (Dangling not compliant)
\item
  Diet

  \begin{itemize}
  \tightlist
  \item
    Low Residue diet
  \item
    Ensure Supplements
  \end{itemize}
\item
  Teaching
\item
  Gum, Mag \& Entereg
\item
  Pain Management

  \begin{itemize}
  \tightlist
  \item
    PCA
  \item
    Tylenol 1gm q6 ATC
  \item
    Gabapentin 300mg TID
  \item
    Oxycodone once at night
  \item
    Tramadol PRN
  \item
    Resume all baseline pain meds
  \end{itemize}
\item
  Home Medications

  \begin{itemize}
  \tightlist
  \item
    Resume all home medications
  \item
    No therapeutic anti-coagiation
  \item
    Diabetes medicines

    \begin{itemize}
    \tightlist
    \item
      Prefer to resume all
    \item
      Sliding-scale insulin if needed
    \end{itemize}
  \end{itemize}
\item
  \emph{Wound Ostomy teaching}
\item
  \emph{CCM for Home Health for stoma care}
\end{itemize}

\textbf{Postop Day 1}

\begin{itemize}
\tightlist
\item
  Labs: K+ and CBC only
\item
  d/c ``pre'' plan \& colon visit
\item
  Heparin lock IV
\item
  Out of bed \textgreater{} 6 hrs
\item
  Diet: low residue
\item
  Pain management

  \begin{enumerate}
  \def\labelenumi{\arabic{enumi})}
  \tightlist
  \item
    d/c PCA
  \item
    Tylenol
  \item
    Gabapentin 300 tid
  \item
    Tramadol
  \item
    Home meds
  \item
    Oxy if lots of pain
  \end{enumerate}
\end{itemize}

\textbf{Postop Day 2}

\begin{itemize}
\tightlist
\item
  No IV fluids unless indicated
\item
  No labs unless indicated
\item
  OOB \textgreater6 hrs
\item
  No PT unless going to rehab or SNF (ask Hill first)
\item
  Pain management

  \begin{itemize}
  \tightlist
  \item
    d/c PCA (unless laparotomy)
  \end{itemize}
\item
  \emph{Wound Ostomy Teaching}
\item
  Foley voiding challenge

  \begin{itemize}
  \tightlist
  \item
    250mL saline into foley -\textgreater{} pull
  \item
    Must void within 2 hours
  \end{itemize}
\item
  Discharge planning

  \begin{itemize}
  \tightlist
  \item
    Consider early D/C
  \end{itemize}
\item
  Otherwise plan

  \begin{itemize}
  \tightlist
  \item
    Patient
  \item
    Nursing
  \item
    \emph{CCM for home health for stoma care}
  \end{itemize}
\item
  Afternoon rounds
\end{itemize}

\begin{enumerate}
\def\labelenumi{\arabic{enumi})}
\tightlist
\item
  Possible home today!
\item
  Check patient 2-3PM
\item
  Ambulation 2x's by PM
\item
  Check for dehydration
\item
  Patient education
\item
  Give estimated date of discharge
\end{enumerate}

\textbf{Postop Day 3-5}
- IVF
- Consider bolus of IV fluids
- Consider maintenance IV if we feel won't resolve soon
- Labs
- No labs unless indicated
- Consider ordering QOD Chem7 if prolonged ileus
- OOB \textgreater6 hrs
- No PT unless likely to go to rehab or SNIF (ask Hill first)
- Pain management
- Same as POD\#1
- Afternoon rounds
1) Possible home today!
2) Check patient 2-3PM
3) Ambulation 2x's by PM
4) Check for dehydration
5) Patient education
6) Give estimated date of discharge

\hypertarget{esophagectomy-postop}{%
\chapter{Esophagectomy Postop}\label{esophagectomy-postop}}

\textbf{ICU Care}

All patients admitted to STICU with Surgical Critical Care Consultation. Surgical critical care will need a phone call immediately after surgery at 6-0366. Listed: Dr Salo, CMC-GI Surgical Oncology, CMC-Surgical Critical Care. Diabetic patients (on insulin preop): Consult Kelli Dunn.

\textbf{Ward Care}

Once stable, patients are transferred to 6T. If a 6T bed is not available, please notify Dr Salo. Historically, over 95\% of patients are transferred to 6T after leaving the ICU.

\textbf{Neuro}

Multimodal pain control:

\begin{itemize}
\tightlist
\item
  gabapentin (300mg tid liquid via Jejunostomy)
\item
  Tylenol (1000mg q6hrs as pediatric liquid via Jejunostomy).
\item
  PCA with subsequent conversion to oxycodone elixir via Jejunostomy.
\item
  No ketorolac (Toradol) given risk of anastomotic failure\hspace{0pt}1\hspace{0pt}
\item
  Home anxiolytics are generally administered at half the home dose
\end{itemize}

\textbf{Cardiovascular}

Postoperative atrial fibrillation is a common occurrence (20\%) after esophagectomy, with the risk increasing in older patients. For patients over age 70, half of patients will develop atrial fibrillation in the postoperative period.

For prevention of atrial fibrillation, beta blockade is used. For patients receiving beta blockers prior to surgery, continuation of beta blockade is recommended. For others, patients are given metoprolol 2.5 mg IV q6hrs which can be titrated up to 10mg IV q6hrs as needed. See STS Guidelines.\hspace{0pt}2\hspace{0pt}

Home anti-hypertensives are usually held in order to allow beta-blockade. Patients who have elevated blood pressures once they are adequately beta-blocked (eg HR 60-70) are usually restarted on their home anti-hypertensives. \emph{Home anti-hypertensives are not routinely restarted postoperatively}

\textbf{Respiratory}

Chest tubes generally consist of a 28Fr Blake drain placed into the right chest. This is placed to water seal when output is less than 200mL/day and there is no leak visible in the Pleurevac container. Chest tubes are usually removed once output is less than 150mL/day and drainage is clear without evidence of chyle (milky appearance). A chest x-ray is obtained after removing a chest tube to look for a pneumothorax.

\textbf{Gastrointestinal}
All patients receive pantoprazole 40mg IV daily and metoclopramide 5-10mg IV q 6hrs

\textbf{Nutrition}

All esophagectomy patients receive a feeding jejunostomy at the time of operation. In the immediate post-operative period, patients receive Osmolite 1.5 starting the day of surgery once they are off pressors. Tube feeds are started at 20mL/hour until flatus and then advanced at 10mL per hour every 8 hours, up to a goal of 60mL per hour (x24 hours). Patients who are on tube feeds prior to surgery are generally restarted on their home tube feed formula.

Patients who do not tolerate Osmolite are switched to Vital 1.5, which is pre-hydrolyzed. In order to allow enough time for switching of tube feedings, patients are generally switched from Vital to their home tube feeding formula 3-4 days prior to discharge. This is typically done when they are transferred out of the ICU.

Obese patients (BMI\textgreater30) are started on Promote at 20mL/hour and increased to a goal of 60mL/hour. Promote contains a more protein relative to carbohydrdates. An alternative is Vital High Protein, which is similar to Promote but using hydrolyzed proteins.

Diarrhea in patients on tube feeds (especially nocturnal diarrhea) needs to be addressed. Despite STICU guidelines for nutrition in trauma patients, diarrhea (especially night-time diarrhea) is justification for alteration in tube feeds. See Diarrhea and Jejunstomy Feeding

\emph{Diabetic Patients}

Patients who require EndoTool in the ICU will need an endocrinology consult. See also \protect\hyperlink{jejunostomy_diabetes}{Jejunostomy Feedings with Diabetes}

\emph{`Free Water'}

In addition to tube feeds, most patients will receive `free' water flushes through the jejunostomy. This is typically done as 240mL four times per day (for a total of 32oz)

\emph{Nasogastric Tubes}

A silicone nasogastric tube (Covidien Salem Sump) is placed in all patients during surgery and the position confirmed by ultrasound intraoperatively. Tubes are positioned so that all 4 dots are outside. Gastric emptying is evaluated with upper GI prior to NG tube removal. Once extubated, upper GI is typically performed on the 2nd through 4th postoperative day. Conversely, patients who are intubated will keep their nasogastric tube until extubated. Radiology ordered as ``upper GI Series''. In the comments section please add ``IsoVue through NG tube. Halo Dr Salo for study''. See Evaluation of GI Function

\emph{Drains}

\begin{itemize}
\tightlist
\item
  JP1: 19Fr Blake drain in left pleura. The exit site the most lateral drain and is secured with a blue suture
\item
  JP2: 19Fr Blake drain in right pleura. The exit site is medial to JP1 and is secured with a black suture
\item
  JP3: 19 Fr Blake drain in abdomen. If used, the exit site is most medial and is secured with a blue suture
\item
  JP4: 15Fr Blake drain in neck (for cervial incision)
\item
  JP5: 15Fr Blake drain in subcutaneous tissue of incision
\end{itemize}

\emph{Evaluation for Anastomotic leak}

\protect\hyperlink{drain_amylase}{Drain amylase} is an inexpensive, specific, and relatively sensitive test for anastomotic leak. Fluid from JP2 (right chest) is sent for ``Body Fluid Amylase'' starting on postoperative day \#4 and continued until postoperative day 9. Drain amylase over 400IU/ml is considered positive and prompts a \protect\hyperlink{ct_esophagram}{CT esophagram} for confirmation. See also \protect\hyperlink{leak_evaluation}{Evaluation for Anastomotic Leak}

\textbf{Renal}
Total fluids (IV + tube feeds) are generally run at 75mL/hour. Foley catheter is removed on the first or second day after surgery. Some patients will need diuresis on the 3rd or 4th postoperative day

\textbf{Heme}
All patients require VTE prophylaxis with Lovenox or heparin SQ.

Preoperative anti-platelet agents are started on the first (aspirin) or second (Plavix) postoperative day if there is no excessive bleeding from the chest tube or JP drains. Patients on preoperative anticoagulation are transitioned to therapeutic Lovenox on the second postoperative day if no signs of bleeding.

\textbf{ID}
Prophylactic Cefazolin and Flagyl are administered for 24 hours and stopped

\hypertarget{esophagectomy-ward}{%
\chapter{Esophagectomy Ward}\label{esophagectomy-ward}}

\textbf{Ward Care}

Once stable, patients are transferred to 6T. If a 6T bed is not available, please notify Dr Salo. Historically, over 95\% of patients are transferred to 6T after leaving the ICU.

\hypertarget{anti-hypertensives}{%
\section{Anti-Hypertensives}\label{anti-hypertensives}}

Patients are transitioned to enteral metoprolol once they are transferred to the ward. Patients on IV metoprolol at 2.5mg IV q6 hours are started on 25mg enteral bid, while patients receiving 5mg IV q6 hours are started on 50mg bid. For patients who are not taking medicines by mouth, liquid metoprolol can be ordered while an inpatient. (Liquid metoprolol is not available as a home medicine.)

Home anti-hypertensives are usually held in order to allow beta-blockade. Patients who have elevated blood pressures once they are adequately beta-blocked (eg HR 60-70) are usually restarted on their home anti-hypertensives. \emph{Home anti-hypertensives are not routinely restarted postoperatively}. Home antihypertensives are restarted on a selective basis

\hypertarget{chest-tubes}{%
\section{Chest tubes}\label{chest-tubes}}

Chest tubes generally consist of a 28Fr Blake drain placed into the right chest. This is placed to water seal when output is less than 200mL/day and there is no leak visible in the Pleurevac container. Chest tubes are usually removed once output is less than 150mL/day and drainage is clear without evidence of chyle (milky appearance). A chest x-ray is obtained after removing a chest tube to look for a pneumothorax.

\hypertarget{gi-medicines}{%
\section{GI Medicines}\label{gi-medicines}}

All patients receive pantoprazole 40mg IV daily and metoclopramide 5-10mg IV q 6hrs. This is later switched to enteric PPI and reglan. For patients \textless age 75, remeron is added as 15mg enteral qhs.

\hypertarget{leak_evaluation}{%
\section{Evaluation for leak}\label{leak_evaluation}}

Experience suggests that the median time to the diagnosis of leak is 7 day after surgery. Currently the risk of anastomotic leak at CMC after transthoracic esophagectomy is 2\%. The most sensitive test for the diagnosis of leak is CT esophagram (see below). Based upon institutional experience, patients are divided into low risk of leak vs high risk for leak based upon drain amylase level and white blood cell count. Patients with a normal drain amylase (400IU/ml) between postoperative day 4 and day 7 AND white blood cell count less than 12 are considered low risk. Patients with either an elevated drain amylase or WBC greater than 12 are evaluated with CT esophagram. In a group of 100 patients, several were found to have elevated drain amylase in the first four days (up to 2000Iu/ml) which subsequently declined, and no evidence of leak was found

\hypertarget{drain_amylase}{%
\subsection{Drain Amylase}\label{drain_amylase}}

JP2 is placed into the right pleura, passes through the hiatus, and is brought out through a trocar site in the medial left upper quadrant. Drain amylase from JP2 is tested beginning on postoperative day \#4 until discharge or postoperative day \#9. JP2 is generally removed prior to patient discharge. JP1 is placed in the left pleura and is generally not tested for amylase.

\hypertarget{ct_esophagram}{%
\subsection{CT esophagram}\label{ct_esophagram}}

This is the most sensitive study for the detection of anastomotic leak. In order to obtain sufficient sensitivity, it requires a pre-contrast scan and the administration of contrast into the esophagus. The need for a pre-contrast scan means that the presence of remnant barium in the esophagus (from a Modified Barium Swallow) makes it more difficult to interpret the scan and should be avoided. Awake patients can drink the contrast. Patients with an NG tube present at the time of the study generally will have contrast administered through their NG tube. Almost all esophagectomy patients will have a Covidien silicone Salem Sump 18Fr tube in place. This has four marks on the tube at 45, 55, 65, and 75cm. The tube is typically positioned with the 4th mark at the nares, which means that the tip is 45cm from the nares and is usually within the gastric conduit AND below the anastomosis. The NG tube also contains side holes which extend 8.5cm above the tip of the tube. The optimal study is done with the NG tube withdrawn so that there are two side-holes above the anastomosis, which means that the tip is approximately 6cm below the anastomosis. A scout CT will be performed, and the radiologist will determine how far back the NG tube needs to be withdrawn. This is communicated to the CT technician, who asks the nurse (or physician) to withdraw the tube the calculated amount (making a note of the starting position of the tube relative to the four marks). This should keep the tip below the level of the anastomosis, so that after the CT scan, the NG tube can be (blindly) advanced back to its original position (at approximately 45cm from the nares).

Patients who clinically deteriorate prior to post-operative day 7 in whom there is high suspicion for a leak (fevers, pleural effusion on chest X-ray, elevated JP amylase, respiratory failure) will undergo a CT esophagram earlier than post-operative day 7.

\hypertarget{anastomotic-leak-treatment}{%
\section{Anastomotic Leak Treatment}\label{anastomotic-leak-treatment}}

If a patient is demonstrated to have a leak, they are made NPO AND will have any pleural effusion treated with a pigtail catheter. Conservative management will generally be successful for most patients with leaks provided 1) there is no evidence of conduit necrosis 2) nutrition is optimized and 3) empyema is treated. Patients with leaks may even need a decortication, which emphasizes the importance of CT scan in the sick post-operative esophagectomy patient, so that an empyema can be diagnosed and treated. Patients with leaks who show signs of systemic illness may need to be considered for an intraluminal stent. Patients who are profoundly ill need to be evaluated for gastric necrosis with upper endoscopy.

\textbf{Nutrition}

All esophagectomy patients receive a feeding jejunostomy at the time of operation. In the immediate post-operative period, patients receive Osmolite 1.5 starting the day of surgery once they are off pressors. Tube feeds are started at 20mL/hour until flatus and then advanced at 10mL per hour every 8 hours, up to a goal of 60mL per hour (x24 hours). Patients who are on tube feeds prior to surgery are generally restarted on their home tube feed formula.

Patients who do not tolerate Osmolite are switched to Vital 1.5, which is pre-hydrolyzed. In order to allow enough time for switching of tube feedings, patients are generally switched from Vital to their home tube feeding formula 3-4 days prior to discharge. This is typically done when they are transferred out of the ICU.

Obese patients (BMI\textgreater30) are started on Promote at 20mL/hour and increased to a goal of 60mL/hour. Promote contains a more protein relative to carbohydrdates. An alternative is Vital High Protein, which is similar to Promote but using hydrolyzed proteins.

Diarrhea in patients on tube feeds (especially nocturnal diarrhea) needs to be addressed. Despite STICU guidelines for nutrition in trauma patients, diarrhea (especially night-time diarrhea) is justification for alteration in tube feeds. See Diarrhea and Jejunstomy Feeding

\emph{Diabetic Patients}

Patients on tube feeds are typically started on continuous (around-the-clock) tube feedings, and subsequently changed to a nocturnal regimen (typically 6pm to 10am). See \protect\hyperlink{jejunostomy_diabetes}{Jejunostomy Feeds in Diabetic Patients}

\emph{`Free Water'}

In addition to tube feeds, most patients will receive `free' water flushes through the jejunostomy. This is typically done as 240mL four times per day (for a total of 32oz)

\emph{Nasogastric Tubes}

A silicone nasogastric tube (Covidien Salem Sump) is placed in all patients during surgery and the position confirmed by ultrasound intraoperatively. Tubes are positioned so that all 4 dots are outside. Gastric emptying is evaluated with upper GI prior to NG tube removal. Once extubated, upper GI is typically performed on the 2nd through 4th postoperative day. Conversely, patients who are intubated will keep their nasogastric tube until extubated.

\emph{Drains}

\begin{itemize}
\tightlist
\item
  JP1: 19Fr Blake drain in left pleura. The exit site the most lateral drain and is secured with a blue suture
\item
  JP2: 19Fr Blake drain in right pleura. The exit site is medial to JP1 and is secured with a black suture
\item
  JP3: 19 Fr Blake drain in abdomen. If used, the exit site is most medial and is secured with a blue suture
\item
  JP4: 15Fr Blake drain in neck (for cervial incision)
\item
  JP5: 15Fr Blake drain in subcutaneous tissue of incision
\end{itemize}

\emph{Evaluation for Anastomotic leak}

Drain amylase is an inexpensive, specific, and relatively sensitive test for anastomotic leak. Fluid from JP2 (right chest) is sent for ``Body Fluid Amylase'' starting on postoperative day \#4 and continued until postoperative day 9. Drain amylase over 400IU/ml is considered positive and prompts a CT esophagram for confirmation. See also Evaluation for Anastomotic Leak

\textbf{Renal}
Total fluids (IV + tube feeds) are generally run at 75mL/hour. Foley catheter is removed on the first or second day after surgery. Some patients will need diuresis on the 3rd or 4th postoperative day

\textbf{Heme}
All patients require VTE prophylaxis with Lovenox or heparin SQ.

Preoperative anti-platelet agents are started on the first (aspirin) or second (Plavix) postoperative day if there is no excessive bleeding from the chest tube or JP drains. Patients on preoperative anticoagulation are transitioned to therapeutic Lovenox on the second postoperative day if no signs of bleeding.

\textbf{ID}
Prophylactic Cefazolin and Flagyl are administered for 24 hours and stopped

Discharge Medicines
Patients after esophagectomy typically go home with the following medicines:

Proton pump inhibitors (will continue for 2 years)
Oxycodone elixir via feeding tube. Current STOP guidelines dictate that patients receive no more than a 7 day supply of opiods at discharage
Reglan 10mg po qid (will stop at 6 weeks post-op)
Remeron 15mg qhs (will continue for 3 months post-op)
Tylenol 1000mg q6 hours as elixir (pediatric form)
Gabapentin 300mg as liquid tid x 14 days
Metoprolol if started postoperatively (most patients). If not on beta blockers preoperatively, this will be cut in half at the first visit and stopped at the second postoperative visit. It is important to limit the use of medicines via jejunostomy in order to lower the risk of clogging the feeding tube. While liquid metoprolol is available for inpatients, it is not available from outpatient pharmacies.

\hypertarget{jejunostomy-feedings}{%
\chapter{Jejunostomy Feedings}\label{jejunostomy-feedings}}

Due to the osmotic load, jejunostomy feedings are given via enteral (Kangaroo) pump rather than bolus feeding. Feedings are generally begun as continuous (around-the-clock) and are then transitioned to nocturnal (generally 6pm to 10am) prior to discharge.

Jejunostomy feedings carry a small but significant risk (\textasciitilde1\%) of small bowel necrosis, as evidenced by the findings of pneumatosis on CT scan and in some cases small bowel necrosis and perforation. The existing literature would suggest that the early symptoms associated with small bowel necrosis are abdominal distension. As a result, patients on jejunostomy tube feeding need to be carefully monitored for distension, and tube feeds held if distension develops.\citep{taylor34} Tube feeds are generally started at 30mL/hour in the immediate postoperative patients. In patients who are awake and in whom it is possible to determine whether or not there are issues of tube feed intolerance, the rate of tube feedings is increased 10mL/hour every 12 hours to a goal of 60mL/hour for women and 75mL/hour for men. In patients who are intubated/sedated, the advancement of tube feeds is individualized, and decisions are made on a daily basis on rounds.

For patients who receive tube feedings preoperatively, the same formula is generally used after surgery.

For patients with BMI less than 35, Osmolite 1.5 is used as a starting formula. If Osmolite is not tolerated due to diarrhea, Vital 1.5 can be used. For patients with BMI greater than 35, Promote is used as a starting formula, as it contains a lower amount of carbohydrates then Osmolite 1.5. Patients who are intolerant to Promote can be switched to Vital High Protein.

Patients less than 150 pounds receive 4 cans of tube feedings, administered at 60mL/hour x 16 hours (6pm to 10am). Patients greater than 150 pounds receive 5 cans of tube feedings, administered at 75mL/hour x 16 hours (6pm to 10am). Patients with BMI \textgreater35 receive 5 cans of Promote (or Vital High Protein).

Glucerna is formulated with carbohydrates with low glycemic index, which is particularly helpful for bolus feeding (via gastrostomy) or patients who are eating. Glycemic index is less important in patients on continuous tube feeds (eg via jejunostomy). In addition, the high fiber context of Glucerna may make this formula more prone to causing clogging of jejunostomy tubes.
Free Water Flushes

Once patients are transferred to the ward, free water via jejunostomy should be ordered as 240mL via jejunostomy qid. This is entered under Flush Enteral Tube for Free Water Requirements.
Diarrhea with Jejunostomy Feedings

Patients who experience diarrhea during jejunostomy administration (especially diarrhea at night) will need this addressed. Several steps

\begin{itemize}
\tightlist
\item
  Send stool for C Diff
\item
  Consider changing tube feedings to a more easily digestible formula. Patients on Osmolite 1.5 can be changed to Vital 1.5. Patients on Vital 1.5 can be changed to Vivonex. Patients on Promote can be changed to Vital High Protein
\item
  Stop tube feedings for 2-4 hours to allow diarrhea to resolve
\item
  restart tube feeds at a lower rate (eg 20mL/hour lower than the prior rate).
\item
  Bannatrol can be given to patients taking an oral diet. Bannatrol is NOT given via jejunostomy tube to avoid clogging.
\item
  Lomotil is generally used as a last resort.
\end{itemize}

\hypertarget{med-administration-via-j-tubes}{%
\section{Med Administration via J-Tubes}\label{med-administration-via-j-tubes}}

Patients who have difficulty with dysphagia or complete esophageal obstruction will need to have their medicines administered via jejunostomy tube. The process of designing a medicine regimen which can safely be administered via enteral tube can be a challenge and may require a consultation with the hospital pharmacist. One the other hand, the financial consequences of a clogged feeding tube are substantial. \textbf{Flomax is never given via jejunostomy tubes due to risk of clogging}.

Several common medicines are available in liquid form:

\begin{itemize}
\tightlist
\item
  Acetaminophen (pediatric formulation)
\item
  Gabapentin
\item
  Oxycodone
\item
  Hydrocodone + acetaminophen
\item
  Reglan
\end{itemize}

\hypertarget{occluded-jejunostomy-tubes}{%
\section{Occluded Jejunostomy Tubes}\label{occluded-jejunostomy-tubes}}

Jejunostomy tubes which are refractory to the usual non-invasive means (warm water, Coca-Cola) will need to be changed over a wire in Interventional Radiology.

\hypertarget{jejunostomy_diabetes}{%
\section{Jejunostomy + Diabetes}\label{jejunostomy_diabetes}}

\textbf{Inpt -- Jejunostomy \textasciitilde{} Diabetes}

Patients on tube feeds are typically started on continuous (around-the-clock) tube feedings, and subsequently changed to a nocturnal regimen (typically 6pm to 10am). The diabetic management for these patients differs depending upon their tube feeding regimen:

\textbf{Diabetics on Continuous Tube Feeds.}

Non-insulin diabetic patients are generally initially given Osmolite 1.5 as it provides higher caloric density and does not contain fiber, which tends to clog the feeding tubes. Diabetic patients requiring insulin can be initially trialed on Osmolite 1.5 but are changed to Promote or Glucerna 1.5 if their blood sugars prove difficult to control (as evidence by either the need for EndoTool or requiring a q6 hour regimen of insulin N + insulin R).

Diabetic patients who need insulin while receiving tube feedings are typically treated initially with continuous tube feedings and around-the-clock insulin. Patients with large insulin requirements may need hourly intravenous insulin (with dosages calculated via EndoTool). Once their insulin requirements are stabilized, they are transitioned to a q6 hour regimen consisting of N and R insulin, typically twice as many units of N insulin as R (for instance, 6Units of N + 3 Units of R insulin every six hours). Patients who require EndoTool or a q6hr regimen need an endocrinology consultation with Dr Kelli Dunn to assist in diabetic management.

\textbf{Diabetics on Nocturnal Tube Feeds}

Most patients are transitioned from continuous tube feeds to nocturnal prior to discharge. Diabetic patients on nocturnal tube feedings typically receive tube feeding from 6pm to 10am and receive insulin at initiation of tube feeds (6pm) and again at 6 hours later (midnight). A typical regimen might be 18U of 70/30 at 1800 and 18U of 70/30 at MN. An alternative might be 12U NPH + 6U Regular at 1800 and 16U NPH and 8U Regular insulin at MN. Because it can take several days to determine the correct insulin regimen, diabetic patients receiving jejunostomy feedings are cycled as early in their hospital course as possible to avoid delaying discharge for blood sugar management.

Diabetic patients receiving insulin will need careful coordination of tube feeding and insulin administration when they are being transitioned from continuous to nocturnal tube feeds. Patients on continuous tube feeds may receive insulin on a q6 hour schedule, while those receiving nocturnal tube feeds receive insulin at 1800 and MN. When patients on continuous tube feeds are transitioned, the tube feeds are stopped at 10am, to be restarted at 6pm that evening. It is critical that as soon as the tube feeds are stopped at 10am, that the q6 hour insulin as stopped as well.

In either case it is critical that if tube feedings are stopped, standing insulin administration (either q6 hour OR 1800 and MN) be stopped as well. In these cases, sliding scale insulin is generally continued.

\begin{longtable}[]{@{}llll@{}}
\toprule
Day & Time & Tube Feeds & Insulin \\
\midrule
\endhead
SUN & MN & 60mL/hr & 8N+4R \\
Mon & 6am & 60mL/hr & 8N+R \\
Mon & Noon & Stop & None \\
Mon & 6pm & 75mL/hr & 16N + 8R \\
Mon & MN & 75mL/hr & 16N+ 8R \\
Tues & 10am & Stop & None \\
Tues & 6pm & 75mL/hr & 16N+ 8R \\
Tues & MN & 75mL/hr & 16N+ 8R \\
Weds & 10am & Stop & None \\
\bottomrule
\end{longtable}

In this example, the patient was receiving 8N + 4R every 6 hours, so total insulin units per day is (8+4) x 4 = 48units. Because the carbohydrate load of 60mL/hour x 24 is roughly equivalent to 75mL/hour x 16 hours, the total insulin administered is roughly the same. When converted to nocturnal dosing, the patient now received 16+8 = 24 units twice (6pm and MN) = 48U. In practice, it may be wiser to begin by adjusting the dose down a little, to perhaps 14U N and 7U R for the first night.

\hypertarget{part-hill-or}{%
\part*{Hill OR}\label{part-hill-or}}
\addcontentsline{toc}{part}{Hill OR}

\hypertarget{colorectal-cases---hill}{%
\chapter{Colorectal Cases - Hill}\label{colorectal-cases---hill}}

\textbf{Pre-op holding}

Please make sure that these ordersets have been placed prior to day of surgery. If inpatient consult, place these orders the morning of surgery in order to minimize confusion.

ADULT SURG Colorectal ERAS MPP Hill This is what has all of the main ERAS components. Tylenol, gabapentin, Decadron, Alvimopan and heparin are all given in pre-op holding

ADULT STANDING Antimicrobial colorectal In general, I will give Ancef to patients with almost all patients with allergies to PCN. They have to remember the ``severe'' reaction. If it is a severe reaction, please use the second line ABX listed in the power plan.

Type and Screen are not typically needed for colectomy. They will have an antibody screen from office. If antibodies present, then d/w attending

\textbf{Intra-op}

Positioning: I typically like to position myself.

Right sided=supine; Left sided=lithotomy; All laparoscopic colon cases will have their arms tucked with a chest tape strap.

NG/OG tubes Not needed. I will have anesthesia place if we have gastric distension. We give a multiple PO meds prior.

Review anesthesia fluid management during time out. 1.6L max volume, urine output not an accurate indicator.

\hypertarget{part-salo-or}{%
\part*{Salo OR}\label{part-salo-or}}
\addcontentsline{toc}{part}{Salo OR}

\hypertarget{cv_port_salo}{%
\chapter{Central Venous Port (IJ)}\label{cv_port_salo}}

\textbf{Room Prep}

\begin{itemize}
\tightlist
\item
  Slider bed (Skytron 3600B) with head section
\item
  C-Arm

  \begin{itemize}
  \tightlist
  \item
    Radiology technician alerted to need for C-Arm
  \item
    Will need lead and thyroid shields for everyone in room
  \end{itemize}
\item
  BK Ultrasound with hockey-stick probe near patient's RIGHT SHOULDER
\end{itemize}

\textbf{Instruments}
- Minor instrument pan

\textbf{Disposables/Meds}

\begin{itemize}
\tightlist
\item
  Confirm choice of port with surgeon. Usual options

  \begin{itemize}
  \tightlist
  \item
    Bard PowerPort VUE with 8Fr attachable catheter (1708062)
  \item
    Bard PowerPort slim Implanted port (for patients with low BMI)
    -Heparin 5mL of 1000U/ml labeled as ``1000 U/ml''
    -Heparin 5mL of 1000U/ml + 45mL saline labeled as ``100 U/ml''
  \item
    Local

    \begin{itemize}
    \tightlist
    \item
      If general anesthesia: Marcaine 0.5\% with epinephrine
    \item
      If MAC: Xylocains 1\% with epinephrine
    \end{itemize}
  \end{itemize}
\item
  1000 drape x3 AND blue paper drapes 4 packs of 2 each = 8 total
\item
  Suture

  \begin{itemize}
  \tightlist
  \item
    3-0 Prolene RB-1 double-arm
  \item
    3-0 Vicryl SH
  \item
    4-0 Monocryl PS2
  \end{itemize}
\end{itemize}

\textbf{Position}

\begin{itemize}
\tightlist
\item
  Supine with left arm tucked, right arm on armboard at side.

  \begin{itemize}
  \tightlist
  \item
    Right arm on armboard in case needed by anesthetist
  \item
    NO shoulder roll
  \end{itemize}
\item
  Foley catheter: usually NOT required -- \emph{check with surgeon}
\item
  Lower body Bair Hugger from abdomen to feet with ONE layer of blankets on top of Bair Hugger. Velcro strap on thighs.
\end{itemize}

\textbf{Prep}

Chloroprep: RIGHT chest, neck to chin and earlobe, shoulder to include deltopectoral groove.

\textbf{Drape}

1000 plastic drapes outline the sterile field for the port. The skin is stretched to avoid a gap between drape and skin. Allow access to the right sternocleidomastoid, right deltopectoral groove, and sternal notch.
Blue paper drapes on top of 1000 drapes
Transverse drape reversed head-to-foot.
Ioban around edges of port field. Skin over SCM is left without Ioban to facilitate ultrasound

\textbf{Preop evaluation}
Allergies
Blood thinners or anti-platelet agents
History of prior central venous lines or ports
History of neck surgery

\textbf{Operation}

\emph{Reverse Trendelenburg}

Port pocket is constructed 1cm below and parallel to clavicle 3cm long. It is essential that there is no bleeding in the pocket (to avoid a port pocket hematoma).

\emph{Trendelenburg}

Right internal jugular vein is identified and its course cephalad-caudad marked on the skin.

Finder needle (22Ga) \emph{OR} micropuncture kit passed into IJ. The needle should enter the vein directly beneath the ultrasound probe.

Skin anesthetized and transverse 8mm counter-incision made at needle entry site

\emph{Respiration held by anesthesia}

16Ga needle passed into IJ under sono. It is essential that the vein is scanned up and down by `rocking' the probe to visualize the tip of the needle as it passes inferior.

J Wire passed through 16Ga needle and needle withdrawn

\emph{Anesthesia resumes respirations}

Ultrasound used to confirm presence of wire within the vein by scanning up and down.

\emph{Level bed}

Fluoroscopy used to confirm position of wire. Dilator and sheath inserted under fluoroscopic visualization (`live'). C-arm backed away off field.

Tunneler connected to tubing on the end with small numbers. Confirm that the collar is in place and place hemostat on the end of the tubing with large numbers (to avoid allowing the collar to fall off). Tunneler bent into a curve to avoid injury to the carotid artery.

Catheter tunneled from port pocket to counter-incision over SCM. The tunneler path describes a gentle arc to avoid kinking the catheter. 1 cm of catheter near tunneler trimmed.

Catheter placed through dilator into central circulation. Most of the catheter is inserted. The catheter will generally not cause arrhythmias.

C-arm brought back onto field.

Peel-away sheath split and removed.

Traction on catheter from the port pocket is used to position catheter approximately 3cm below carina. It may be necessary to `orbit' the C-arm if the catheter overlies the spine.

Port pocket is measured and catheter trimmed (after sliding collar superior) and attached to port. It is essential that the catheter come to rest within 1mm of the port before locking the collar in place. In order to avoid pulling on the catheter (and changing the position of the catheter tip) the port is rotated (not the catheter). Collar is locked in place.

Port accessed with straight Huber needle with 100U/ml heparinized saline. Blood is withdrawn into port. Needle is left in the port and the syringe detached. Syringe with concentrated flush (1000U/ml) is attached to the needle and the port flushed (without aspiration of blood). Syringe and needle are removed.

Port sutured to the underlying pectoralis fascia with 2 sutures of 3-0 Prolene, one forehand and one backhand. Sutures are tied and cut.

The port pocket irrigated and the incision closed with subcutaneous 3-0 Vicryl followed by subcuticular 4-0 Monocryl. The skin is dressed with Dermabond.

\textbf{Postop Orders}

CXR in recovery to confirm central line placement

\hypertarget{lap-jejunostomy}{%
\chapter{Lap Jejunostomy}\label{lap-jejunostomy}}

\textbf{Room Prep}

\begin{itemize}
\tightlist
\item
  EGD cart near patient's LEFT SHOULDER (with ADULT EGD scope)
\item
  If central venous port is placed at the same time:
\item
  Slider bed (Skytron 3600B) with head section
\item
  Radiology technician alerted to need for C-Arm
\item
  BK Ultrasound with hockey-stick probe near patient's RIGHT SHOULDER
\end{itemize}

\textbf{Instruments}

\begin{itemize}
\tightlist
\item
  5mm 30 degree scope AND 5mm 0 degree scope
\item
  SRI laparoscopic Pan
\item
  Salo laparoscopic instruments
\end{itemize}

\textbf{Disposables/Meds}

\begin{itemize}
\tightlist
\item
  Veress needle (with 10mL syringe and saline)
\item
  5mm Z-thread optical port (3 on table, 2 more in room)
\item
  Transverse drape AND laparoscopy drape
\item
  Confirm choice of port with surgeon. Usual options

  \begin{itemize}
  \tightlist
  \item
    Bard PowerPort 8Fr xx8062
  \item
    Bard PowerPort 8Fr xx8000 (low profile)
  \end{itemize}
\item
  Heparinized saline: 100U/ml (dilute) and 1000U/ml (concentrated)
\item
  1000 drape x3 AND blue paper drapes 4 packs of 2 each = 8 total
\item
  Micropuncture kit available/not open (from Anesthesia)
\item
  Jejunostomy tube: MIC 0301-14
\item
  Silk 2-0 on RB-1 needle (on Surgical Oncology suture cart)
\end{itemize}

\textbf{Position}

\begin{itemize}
\tightlist
\item
  Supine with left arm tucked, right arm on armboard at side.
\item
  Foley catheter: Usually required -- \emph{check with surgeon}
\item
  Lower body Bair Hugger on thighs. ONE layer of blankets on top of Bair Hugger. Velcro strap on thighs. NO PILLOW UNDERNEATH LEGS.
\end{itemize}

\textbf{Prep}

Chloroprep (two sticks) of abdomen (need to keep pubis in field, as well as right anterior superior iliac spine), both costal margins.

If port: RIGHT chest, neck to chin and earlobe, shoulder to include deltopectoral groove

\textbf{Drape}

If central venous port: Perimeter of field draped with 1000 (clear adhesive) drapes. Four 1000 drapes around port site:

\begin{itemize}
\tightlist
\item
  Medial border: From Angle of Louis superiorly along midline to chin.
\item
  Superior border: Inferior to jaw (to allow access to right internal jugular vein and SCM)
\item
  Laterally: From inferior to ear down to right shoulder
\item
  Inferior: From lateral shoulder medially to Angle of Louis
\end{itemize}

Abdomen: Two 1000 drapes used inferiorly keeping pubis and right anterior inferior iliac spine in field. This is critical as the far inferior/lateral RLQ needs to be in the field for optimal port placement.

Six Blue Paper Drapes around perimeter of field (on top of 1000 drapes)

If central venous port: Transverse sheet TURNED HEAD-TO-FOOT turned at an angle to keep deltopectoral groove and SCM within the field

Laparoscopy drape skewed to inferior and right to keep pubis and right ASIS in the field.

Turn on Bair Hugger only AFTER drapes in place

\textbf{Indications}

Laparoscopic jejunostomy is used for enteral nutrition in patients prior to planned (or possible) esophagectomy or gastrectomy or those for whom the stomach is otherwise not available (ie after esophagectomy or gastrectomy). Patients with metastatic esophageal cancer who need enteral access are generally treated with a gastrostomy, which does not require feeding with a pump

Preop (Resident)
Preop orderset: search for ``Jejunostomy''

Review Clinical Information (Resident)
Review staging scans (especially PET scan) to identify suspicious areas on imaging which need to be investigated at the time of laparoscopy
Outpatient anticoagulation use (warfarin, Xaralto, aspirin, Plavix)
Review dietitian's recommendations (how many cans of feeding per day?)
If patient is scheduled for central venous port
Confirm that a port has not already been placed
Prior history of central venous lines?
Confirm location of port placement with surgeon (left vs right)

\textbf{Operation}

If a central venous port is placed, the port is performed first. See \protect\hyperlink{cv_port_salo}{IJ Port}

Abdominal access is obtained in one of two ways:

Infraumbilical approach using modified Hasson technique. If the peritoneum is not easily entered, a Veress needle is used to insufflate, followed by incision of the fascia with a 15 blade, and a 5mm optical port (Applied Medical Kii Fios First Entry Z-Thread Trocar)
Veress needle inserted in LEFT upper quadrant just inferior to costal margin. Abdominal entry with 5mm optical Z-Thread port.
5mm port in right upper quadrant, 5mm port in RLQ just lateral to rectus, 5mm camera port between RUQ and RLQ ports

The transverse colon is now elevated (using the umbilical port, if used) and the ligament of Treitz is identified. The proximal bowel is arranged in a ``C'' configuration to confirm the proximal and distal ends of bowel.

A site for placement of the jejunostomy is selected on the skin, left lateral and just superior to the umbilicus. A site is selected on the bowel in the most proximal site on the jejunostomy selected which would allow for placement of the jejunostomy without tension, but at least 20cm from ligament of Treitz.

The proximal jejunum is sutured to the anterior abdominal wall with 2-0 silk. This is usually done with a 9″ suture which is introduced into the abdomen with a needle driver ``Korean Style'' or ``Paraguayan Style.'' Two cm distal to this suture, a diamond of sutures is placed around the proposed tube site, and one suture placed distal to avoid torsion. The final arrangement of sutures is one proximal and one distal and 4-6 sutures around the tube. All sutures were marked with hemostatic clips to facilitate replacement of the tube via fluoroscopy should the tube become dislodged.

Using Seldinger technique, a 16Fr Cook catheter introducer kit is placed within the jejunum.

A MIC 14Fr jejunostomy tube (0301-14) with the tabs trimmed with a scalpel, is inserted through the sheath and positioned in the jejunum. The tube was secured with a suture of 0 silk.

If a balloon tube is used, an 18Fr Cook dilater and sheath is used, followed by a MIC 14Fr jejunustomy tube (0200-14) and the balloon inflated with 7mL of sterile WATER.

The tube is secured with 0 silk and dressed with a BioPatch and a Tagederm dressing.

The abdomen is desufflated and the port sites closed with 4-0 Monocryl, followed by dermabond.

After dressings are applied, a Lopez valve is attached with the long Christmas-tree end placed into the jejunostomy tube.

Endoscopy
The scope is set up:

Suction and aspiration valves inserted and working
Suction tubing attached
Biopsy valve attached and not leaking
Cart set up for recording by powering on the Stryker SDC digital capture box
A bite block is used and the scope lubricated. A neonatal scope may be necessary in patients with a tight stricture. Important findings to record:

Level in cm from the incisors, of the most proximal area of Barrett's esophagus.
Level in cm from the incisors of the GE junction
Appearance of the GE junction on retroflexed view. Extent of invasion of the tumor into the cardia or fundus.
The scope is withdrawn and the hypopharynx suctioned. The liquid from the `First Step' disinfectant is suctioned through the scope, followed by water.

\hypertarget{lap_gastrostomy_salo}{%
\chapter{Lap Gastrostomy}\label{lap_gastrostomy_salo}}

\textbf{Room Prep}

\begin{itemize}
\tightlist
\item
  EGD cart near patient's LEFT SHOULDER (with NEONATAL EGD scope)
\item
  If central venous port is placed at the same time:

  \begin{itemize}
  \tightlist
  \item
    Slider bed (Skytron 3600B) with head section
  \item
    Radiology technician alerted to need for C-Arm
  \item
    BK Ultrasound with hockey-stick probe near patient's RIGHT SHOULDER
  \end{itemize}
\end{itemize}

\textbf{Instruments}

\begin{itemize}
\tightlist
\item
  5mm 30 degree scope AND 5mm 0 degree scope
\item
  SRI laparoscopic Pan (available)
\end{itemize}

\textbf{Disposables/Meds}

\begin{itemize}
\tightlist
\item
  Veress needle (with 10mL syringe and saline)
\item
  5mm Z-thread optical port (3 more in room)
\item
  Transverse drape AND laparoscopy drape
\item
  Confirm choice of port with surgeon. Usual options

  \begin{itemize}
  \tightlist
  \item
    Bard PowerPort 8Fr xx8062
  \item
    Bard PowerPort 8Fr xx8000 (low profile)
  \end{itemize}
\item
  Heparinized saline: 100U/ml (dilute) and 1000U/ml (concentrated)
\item
  1000 drape x4 AND blue plastic adhesive drapes 4 packs of 2 each = 8 total
\item
  Micropuncture kit available/not open (from Anesthesia)
\item
  20Fr Laparoscopic gastrostomy kit in room/not open
\item
  16Fr MIC gastrostomy tube in room/not open
\item
  Gastrostomy 20Fr Pull PEG (in vending machine)
\item
  GI Anchors (``T-fasteners'') in room/not open
\end{itemize}

\textbf{Endoscope Setup -- Neonatal EGD scope}

\begin{itemize}
\tightlist
\item
  Valves attached and working (suction/aspiration/biopsy cap)
\item
  Suction tubing attached
\item
  Connect water bottle to left-hand port
\item
  Gauze sponges, lubricant, plastic ``tray'' from gauze filled with water
\item
  \textbf{Yellow} (small) bite block
\item
  ``First step'' sanitizer
\end{itemize}

\textbf{Anesthesia}

\begin{itemize}
\tightlist
\item
  ET Tube taped to left. Head turned to the left on donut
\item
  No EKG electrodes on anterior right chest
\end{itemize}

\textbf{Position}

\begin{itemize}
\tightlist
\item
  Supine with left arm tucked, right arm on armboard at side.
\item
  Foley catheter: usually NOT required -- \emph{check with surgeon}
\item
  Lower body Bair Hugger on thighs. ONE layer of blankets on top of Bair Hugger. Velcro strap on thighs. NO PILLOW UNDERNEATH LEGS.
\end{itemize}

\textbf{Prep}

Chloroprep (two sticks) of abdomen (need to keep pubis in field, as well as right anterior superior iliac spine), both costal margins.

If port: RIGHT chest, neck to chin and earlobe, shoulder to include deltopectoral groove

\textbf{Indications}

Laparoscopic gastrostomy is used in patients with esophageal obstruction . Gastrostomy feedings are much easier than jejunostomy, as they can be administered via syringe or gravity bag. By contrast, jejunostomy feedings require administration via pump. Gastrostomy is usually done as an outpatient unless there are concerns for refeeding.

\textbf{Review Clinical Information (Resident)}

Review staging scans (especially PET scan) to identify suspicious areas on imaging which need to be investigated at the time of laparoscopy
Outpatient anticoagulation use (warfarin, Xaralto, aspirin, Plavix)
Review dietitian's recommendations (how many cartons per day?)

\textbf{Drape}

If central venous port: Perimeter of field draped with 1000 (clear adhesive) drapes. Four 1000 drapes around port site:

\begin{itemize}
\tightlist
\item
  Medial border: From Angle of Louis superiorly along midline to chin.
\item
  Superior border: Inferior to jaw (to allow access to right internal jugular vein and SCM)
\item
  Laterally: From inferior to ear down to right shoulder
\item
  Inferior: From lateral shoulder medially to Angle of Louis
\end{itemize}

Abdomen: Blue adhesive drapes used inferiorly keeping pubis and right anterior inferior iliac spine in field. This is critical as the far inferior/lateral RLQ needs to be in the field for optimal port placement if the patient needs a jejunostomy.

Six Blue Adhesive Drapes around perimeter of field (on top of 1000 drapes)

If central venous port: Transverse sheet TURNED HEAD-TO-FOOT turned at an angle to keep deltopectoral groove and SCM within the field

Laparoscopy drape skewed to inferior and right to keep pubis and right ASIS in the field.

Turn on Bair Hugger only AFTER drapes in place

\textbf{Operation}

If a central venous port is placed, the port is performed first. See \protect\hyperlink{cv_port_salo}{IJ Port}

\textbf{Abdominal access} is obtained in one of two ways:

\begin{itemize}
\tightlist
\item
  Veress needle inserted in left upper quadrant just inferior to costal margin. Abdominal entry with 5mm optical port at lateral border of rectus just superior to umbilicus
\item
  Infraumbilical approach using modified Hasson technique. If the peritoneum is not easily entered, a Veress needle is used to insufflate, followed by incision of the fascia with a 15 blade, and a 5mm optical port
\end{itemize}

The insufflation pressure is decreased to 4mmHg and the abdomen vented to drop the pressure. A 30 degree scope is passed inferior to the falciform ligament into the left upper quadrant over the lateral segment of liver. The post of the scope is positioned to the left, allowing visualization of the lesser curvature of the stomach with the end of the scope near the left aspect of the falciform.

A site for placement of the gastrostomy is selected on the skin, using a 22Ga needle as a finder. The site is marked, then infiltrated with local anesthetic and a 5mm transverse incision made.

Endoscopy is performed and the following noted:

Level in cm from the incisors, of the most proximal area of Barrett's esophagus.
Level in cm from the incisors of the GE junction
Appearance of the GE junction on retroflexed view. Extent of invasion of the tumor into the cardia or fundus.

A bite block is positioned (unless the patient is edentulous). The endoscope (usually a neonatal scope) is introduced into the esophagus and the video capture started. If the tumor will not allow passage of the scope, do not force the scope.

Once the scope is passed into the stomach, the fundus and duodenal bulb are suctioned.

Insufflation is then reduced to 4mm and the stomach insufflated to find the optimal location for tube placement which will minimize tension and will avoid injury to the right gastroepiploic artery. The reduced laparoscopic insufflation pressure allows endoscopic insufflation of the stomach.

Gastrostomy tube placement is done either by Pull or Seldinger technique.

\textbf{Seldinger Gastrostomy}

Four T-fasteners are then used to affix the stomach to the anterior abdominal wall. These are arranged at 8:00, 10:00, 2:00, 4:00 relative to the proposed tube site. T-fasteners are not placed inferior to the tube site to avoid injury to right gastroepiploic vessels.

The J wire is then passed into the stomach, followed by the dilators, up to 20Fr. A 16Fr MIC gastrostomy tube is introduced and the balloon inflated with 5mL of \emph{water}.

\textbf{Pull Gastrostomy}

A 20Fr PULL PEG tube kit is opened. The snare and tube are passed to the upper operator. The snare is passed through the scope and opened in anticipation of the passage of the wire.

The angiocath from the kit is placed into the stomach through the abdominal wall. Once the snare has grasped the angiocath, the needle is withdrawn and the split wire passed through the Angiocath into the stomach. The snare is adjusted to grasp the split wire, which is pulled out through the mouth.

The laparoscopic port site is considered clean. The PEG tube, once it is pulled through the mouth, is considered dirty. The right abdomen is covered with a towel to protect the laparoscopic site.

The recording is now stopped. The split wire is joined to the PEG tube, which is pulled into place by the abdominal operator. The tapered portion of the tube will be the first source of resistance, which may require firm traction. The split wire (and PEG tube) is dropped of the table to the patient's left.

Once the tapered portion of the tube is through the abdominal wall skin, the next point of resistance will be the bumper of the PEG tube passing through the tumor. In general, if a 5mm neonatal scope can pass the tumor, a 20Fr PEG tube can pass as well.

The tube is pulled into position and the measurement at the skin noted. The stomach is aspirated by the upper operator and insufflation is resumed at 8mmHg. Tension on the PEG tube is adjusted to allow apposition of the gastric serosa to the abdominal wall. If the tube is not easily apposed to the abdominal wall, T fasteners (``GI Anchors'') must be employed.

The scope is withdrawn and the hypopharynx suctioned. The liquid from the `First Step' disinfectant is suctioned through the scope, followed by water.

The abdomen is desufflated and the port sites closed with 4-0 Monocryl.

\hypertarget{esophagectomy-1-stage}{%
\chapter{Esophagectomy 1 Stage}\label{esophagectomy-1-stage}}

\hypertarget{indications}{%
\section{Indications}\label{indications}}

MI Ivor Lewis (``One Stage'') esophagectomy is the most common approach to esophagectomy. The patient is positioned in `corkscrew' position to allow simultaneous access to the abdomen and chest by prepping both abdomen and chest.

\hypertarget{room-prep}{%
\section{Room Prep}\label{room-prep}}

Confirm positioning equipment (Salo positioner `bucket'):

\begin{itemize}
\tightlist
\item
  Four black side-rail clamps
\item
  Four rectangular lateral positioners
\item
  Yassargil socket OR large-bore Clark socket
\item
  Well-leg holder (used as an arm holder)
\end{itemize}

Critical supplies (chek prior to pt in room)

\begin{itemize}
\tightlist
\item
  Stapler 25mm DSTXL
\item
  Orvil 25mm
\item
  Stapler 21mm DSTXL
\item
  Orvil 21mm
\item
  Echelon 60mm stapler with 10 gold loads and 2 gray loads
\item
  Gel Port
\end{itemize}

Review inital PET images

\textbf{Anesthesia}

\begin{itemize}
\tightlist
\item
  Dual-lumen ETT tube (taped to left)
\item
  Arterial line (left arm will be on arm board)
\item
  Anesthesia may place paraspinal muscle blocks on right side.
\end{itemize}

\#\#Position

Supine on blue foam pad.

Mark upper midline with non-sterile skin marker

Foley catheter (with Criticor temperature sensor)

Bovie pad

Lower-body Bair Hugger at level of thighs

Hair clipped from abdomen, right chest, and right axilla.

Pre-existing jejunostomy

\begin{itemize}
\tightlist
\item
  Prep into field
\item
  Remove any eschar at site
\item
  Secure to skin inferiorly with 0 silk
\end{itemize}

Shoulders are shifted to the right in preparation for `corkscrew' positioning

Lateral positioners positioned with pad extending from greater trochanter inferiorly.

Velcro strap over thighs and over lateral positioners

Yassargil socket attached to headpiece of bed on left side.

Well-leg holder attached to Yassargil socket and used to support right forearm (which will cross body). Right arm crosses body and is supported on well-leg holder.

Lateral positioner placed posterior to spine and scapula to support right chest, allowing access for thoracoscopy.

Arm holder is dropped towards the floor enough that right arm is brought forward.

\textbf{Prep}

Chloroprep (two sticks) of abdomen, right chest, right axilla. Particular attention to prepping as far as possible to the left lateral side and the right lateral side. Nipple prepped into field.

\textbf{Drape}

Proximal right arm draped with 1000 (clear adhesive) drape.

Two blue plastic ``U'' drapes with center of the ``U'' on either lateral side with tails forming the perimeter of the field Trauma drape. All of right chest and axilla is kept within the field, as is the lateral aspect of the left upper quadrant. The field does not need to extend inferior to the umbilicus. In general, it is usually possible to keep all of these area in the field without cutting the drape, except in very large patients. Ioban strips (4'') aound the periphery of the surgical field after the trauma drape. Laparoscopic cords (gas, light cord, camera) to tower at left shoulder. Suction irrigator brought off field. Laparoscopic LigaSure (2 bars) and Bovie (30/30).

\hypertarget{time-out}{%
\subsection{Time Out}\label{time-out}}

Operation header on consent

Tumor location and likelihood of division of esophagus from abdomen (two-phase operation) vs division of esophagus from chest (four-phage operation).

Blood:

\begin{itemize}
\tightlist
\item
  Surgeon's expectation of blood loss
\item
  Availability of blood (type/screen vs type/cross).
\end{itemize}

\emph{Comorbidities:}

Cardiopulmonary disease. If echo, ejection fraction and aortic valve area (if abnormal)

Beta blockage: Note whether patient on home beta blockade and if so, whether home medication was taken the morning of surgery.

Anticipated Intraop Problems:

\begin{itemize}
\tightlist
\item
  Possibility of tension left pneumothorax due to carbon dioxide entry into left chest during mediastinal dissection
\item
  Expectation of ventilatory difficulties due to carbon dioxide entry into the right chest Gastric mobilization -- Greater Curvature
\end{itemize}

\hypertarget{gastric-mobilization}{%
\section{Gastric Mobilization}\label{gastric-mobilization}}

After infiltration of 1\% marcaine, 7cm upper midline incision made for handport. Small incision is made and pylorus palpated, if possible. Incision is centered over pylorus. GelPort inserted, and abdomen insufflated to 15mmHg. Two 5mm ports placed LUQ. Medial LUQ 5mm port placed in angle between left costal margin and superior edge of GelPort ring. Lateral LUQ 5mm port placed as far lateral as possible. Depending upon visualization, a third port may be required between these two and somewhat more inferior.

If feasible, division of gastrocolic omentum starts by delivering transverse colon into GelPort and dividing ligament with cautery and LigaSure in the avascular plane just cephalad to transverse colon. Dissection proceeds as far proximal and distal and feasible. It is important to avoid damage to the right gastroepiploic artery.

Colon is returned to abdomen and gastricolic ligament divided going using LigaSure, taking care to avoid the colon and the right gastroepiploic artery. For patients with a bulky omentum, it may be helpful to place an additional port in the left mid-quadrant for the camera to facilitate dissection of omentum off the transverse colon. Stomach is retracted to the patient's right with the back of the left hand, placing the gastrocolic ligament (and short gastric arteries) on stretch. Left gastroepippoic artery divided with LigaSure near its origin. Short gastric arteries divided with Ligasure close to spleen. As superior aspect of short gastrics is reached, the dissection plane shifts medially to create a tunnel towards the base of the left crus. This places the most superior short gastric vessels on stretch and facilitates their division. Once all short gastric vessels divided, peritoneum tethering fundus to diaphragm is incised, and fundus brushed medially.

\hypertarget{distal-mobilization}{%
\subsection{Distal mobilization}\label{distal-mobilization}}

Attention is directed to mobilizing the lateral aspect of the duodenum. This can either be accomplished with the camera through a LUQ port and a 45 degree camera or by placing an additional 5mm port in the RLQ. Hook cautery is used to incise the connective tissue lateral and posterior to the duodenum. The gastrocolic ligament is now dissected distally, taking care to preserve the integrity of the right gastroepiploic vessels. An areolar plane generally exists between the fat pad containing the right gastroepiploic vessels and that containing the transverse mesocolon vessels.

\hypertarget{left-gastric-artery}{%
\subsection{Left gastric artery}\label{left-gastric-artery}}

Lymph nodes around the celiac axis and left gastric artery are now dissected. The extend of dissection depends upon the tumor location and the presence of nodes her either based upon imaging or palpation. Dissection begins on the superior edge of the pancreas, and proceeds superiorly to the right crus. The left gastric (coronary) vein is usually located to the left of the arery and is divided with the LigaSure. In a two-phase approach, the left gastric artery is now divided with a 30mm gray load (2.0mm) Echelon linear stapler. For patients with mid-esophageal tumors, (for whom a four-phase approach is used), if there is any question about the resectability of the tumor, division of the left gastric gastric artery is generally deferred until the second abdominal phase.

\hypertarget{mediastinal-dissection}{%
\subsection{Mediastinal dissection}\label{mediastinal-dissection}}

The esophagus is now dissected circumferentially at the gastroesophageal junction. The peritoneum overlying the diaphragm is incised, and circumferential dissection of esopahgus performed. On the left side, it is helpful to distract the left crus laterally with a Prestige clamp placed on the left crus The right pleura is widely entered in otder to both facilitate the thoracic dissection of the esophagus and placing the conduit into the right chest in preparation for the final thoracic phase. Division of Esophagus (Two-Phase only) In a four-phase approach, the esophagus is divided from the right chest during the first (of two) thoracic phases. In a two-phase approach, the esophagus is divided from the abdomen by reaching up into the mediastinum to divide the esophagus above the tumor. This is only feasible for tumors of the gastroesophageal junction. In a two-phase approach, the esophagus is divided with a 60mm Medium-Thich (Gold) Echelon TriStaple stapler.

\hypertarget{division-of-esophagus-four-phase}{%
\subsection{Division of Esophagus (Four-Phase)}\label{division-of-esophagus-four-phase}}

Penrose Drain (Four-Phase only) In a four-phase approach, the esophagus is divided from the chest. In order to facilitate the thoracic dissection, a 1/4″ penrose drain is tied around the distal esophagus and the drain is slid cephalad into the mediastinum. The `tails' of the drain are directed into the right chest so that they can be grasped from the right chest during the thoracic phase and can be used to provide traction on the esophagus

\hypertarget{entry-into-right-chest}{%
\subsection{Entry into right chest}\label{entry-into-right-chest}}

The pneumoperitoneum in the abdomen is vented and the gas pressure turned down to 8mmHg in preparation for the thoracic phase. The bed is rotated to the left 20 degrees. A site for entry in to the left chest is selected just posterior to the tip of the scapula. An incision is made here and a 5mm optical port with a 5mm 0 degree scope used to enter the chest. The chest is insufflated at 8mmHg of carbon dioxide, which helps to both copllapase the lung and depress the diaphragm. Two 12mm ports are placed. The more superior is placed lateral and superior to the nipple. The inferior port is placed just lateral to the diaphragmatic reflection. It is ciritcal to avoid injury to the diaphragm (and liver) with the inferior port placement. A mini-thoracotomy incision is placed along the mid-axillary line, frequently in the same interspace as the inferior/anterior 12mm port. The chest is entered just superior to the rib and the intercostal muscles divided with the LigaSure device to allow the ribs to separate. A narrow Deaver retractor is used to guage the space between the ribs, as the width of the retractor approximates the diameter of the 25mm stapler. A 5mm `U' port is generally placed as high as possible midway between the scapular tip port and the anterior/superior 12mm port t0o the 4 right 5

Division of Esophagus (Two-Phase)
In patients with low-lying tumors, it is possible to divide the esophagus above the tumor from the abdominal approach. This evaluation is facilitated by review of the properative endoscopy (and EUS), and the PET scan, particularly the PET obtained prior to neoadjuvant chemoradiation.

After dissection of the mediastinum, a Echelon stapler with a 60mm Medium-Thick (Gold) load is inserted through a 12mm port either placed either through the GelPort or by upsizing the most lateral LUQ

Construction of Conduit
The distal esophagus and stomach is exteriorized through the GelPort. The lesser curvature vessels are divided with the LigaSure 7-9cm cephalad from the pylorus. The stomach is now placed on stretch along the greater curvature. An Echelon Medium-Thick (Gold) stapler is used to construct a 5-6cm wide gastric tube. In constructing the conduit in a patient with a tumor of the GE jnction invading into the cardia, it is important to be certain that the staple line to construct the conduit stays clear of the tumor. Patients with tumors invading the cardia are at risk of a positive distal margin, meaning that microscopic tumor may be left in the wall of the gastric conduit. To make things more complicated, in patients with low-lying esophageal or GE junction tumors, not all of the length of the conduit are needed in order to reach to the level of the esophageal transection. After construction of the anastomosis, the `extra' cephalad portion of the conduit (near the angle of His) are excised as the `additional gastric margin.' In order to distinguish which portions of the conduit staple line which will be used to replace the esophagus and those which will be included in the `additional gastric margin', both sides of the gastric conduit staple line are marked with sutures designated `A', `B', `C', etc proceeding from the Angle of His to the antrum.

The distal esophagus and GE junction are now sent for frozen section.

Feeding jejunostomy
The ligament of Trietz is identified and the jejunum is identified 20cm distal and marked with a directional suture. A site is selected to the left of the handport incision. A 16Fr Cook Introducer Kit is used to pass a 16Ga needle, followed by a J wire, through the left rectus muscle. A skin incision 4mm in length is made adjacent to the J wire. The 16Fr dilator and sheath are now passed through the rectus muscle and the dilator and wire removed. A 14Fr Jejunostomy Tube 0301-14 is selected and the `wings' trimmed off with a scalpel. The jejunostomy tube is passed through the peelaway sheath, which is removed.

A pursestring suture 1.5cm in diameter is placed on the antimesenteric border of the jejunum. The pursestring is started and ends on the lateral aspect. A second 16Fr Cook Introduced Kit is used to introduce the J wire through the center of the pursestring.

Placement of Drains
Two (or three) 19Fr full-fluted Blake drains (72230) are placed:

JP1: placed into the left pleura through the hiatus. Drain is brought out through the most lateral 5mm port site on the LUQ
JP2: placed into the right pleura through the hiatus. Drain is brought out through the next most medial 5mm LUQ port site
JP3: (optional) placed in the abdomen posterior to the left lateral segment of the liver and brought out through the most medial 5mm LUQ port site
Transposition of Conduit
The gastric conduit is now placed into the right pleura through the mediastinum, with the assistance of a laparoscopic Babcock and gentle pressure on the greater curvature with the fingers.

Entrance into the R chest (Two Phase)
For two-phase operations, ports are now placed into the right chest, as stated above . In similar fashion, the inferior pulmonary ligament in dissected and the lung reflected anterior.

Anastomosis
The right chest is entered and the right lung reflected anterior with the paddle placed in the superior/anterior port. The gastric conduit is placed into the mediastinum by tucking it medially from the right pleura into the posterior mediastinum, in order to allow the conduit to take the most direct path from the hiatus to the proximal esophagus. Gentle superior tension is now applied to the conduit in order to eliminate redundancy. The paddle retractor is now moved from the anterior/inferior port to the anterior/superior 12mm port and the lung reflected anterior and inferior.

OrVil
The OrVil device is now used to place a EEA anvil into the distal esophagus. In general, a 25mm size is selected, unless the patient has particularly small frame, in which case a 21mm size is used. Two stay sutures of 2-0 silk on RB-1 needles are placed in the center of the esophageal staple line 3mm apart. A Harmonic scalpel is used to divide the staple line. The OrVil device is passed through the mouth and is passed through the fenestration in the staple line. The anvil portion of the OrVil is oriented so that the rounded portion is placed against the roof of the mouth. The OrVil is guided into the hypopharynx by pulling on the the tube end of the device. As the anvil approaches the hypopharynx, the jaw is pulled forward to allow passage of the anvil.

The shaft of the anvil is brought through the esophageal staple line, and the tube disconnected from the anvil by cutting the blue sutures.

The superior end of the conduit is opened along the staple line and the DST XL stapler (matching the diameter of the OrVil) introduced through the Alexis device. The stapler shaft is placed into the open end of the gastric conduit and the conduit pulled over it (`sock over shoe'). The stapler spike is brought out through the greater curvature. The anvil is grasped with a Maryland grasper placed through a superior 5mm port and the two components of the stapler are mated and the stapler tightened and fired. The knob of the stapler is rotated two turns counter-clockwise until a click is felt, at which time the anvil will flip. The stapler is withdrawn and the donuts examined and sent for pathologic exam.

The anastomosis is completed by firing a Echelon Medium-Thick (Gold) linear) across the conduit cephalad to the anastomosis. The excess conduit is sent for pathologic exam as `additional gastric margin.'

NG Tube
A Covidien Salem Sump 18Fr nasogastric tube is passed by the anesthetist. A laparoscopic BK ultrasound is used to monitor the passage of the NG tube through the esophagus and into the gastric conduit. The NG tube is passed to the level that all for dots are outside the nose, with the 4th dot at the nares. The NG tube is secured with an AMT bridle.

Chest Tube
A 28Fr Blake chest tube is placed through the anterior/inferior 12mm port and is positioned into the posterior mediastinum. JP2 is placed near the gastric conduit. The right lung is re-inflated.

Closure
The stapler access port incision is closed with 0 Vicryl to approximate the serratus muscle. The incisions are closed with 4-0 Monocryl followed by Dermabond.

\hypertarget{part-esophageal-cancer}{%
\part*{Esophageal Cancer}\label{part-esophageal-cancer}}
\addcontentsline{toc}{part}{Esophageal Cancer}

\hypertarget{EsoIntro}{%
\chapter{Esophageal Overview}\label{EsoIntro}}

Esophageal cancers can be grouped into 4 treatment categories:

\begin{itemize}
\tightlist
\item
  \protect\hyperlink{superficial}{Superficial} \(\rightarrow\) Endoscopic therapy
\item
  \protect\hyperlink{localized}{Localized} \(\rightarrow\) Primary surgery
\item
  \protect\hyperlink{locally_advanced}{Locally Advanced} \(\rightarrow\) Trimodality therapy
\item
  \protect\hyperlink{metastatic}{Metastatic} \(\rightarrow\) Systemic therapy
\end{itemize}

Patients with minimal dysphagia, no weight loss, and small (\textless3cm length) tumors are evaluated with endoscopic ultrasound:

\begin{itemize}
\tightlist
\item
  If uT1 on EUS and \textless2cm in size, \protect\hyperlink{emr}{endoscopic mucosal resection} yields more information and may be therapeutic for tumors with negative margins and without high-risk features.
\item
  If uT2N0 on EUS, and PET scan shows a small tumor (MTV \textless10cm\textsuperscript{3}), \protect\hyperlink{primary_surgery}{primary surgery} is preferred in patients who are good surgical risks
\item
  If T3 or N+ on EUS, if PET shows no metastatic disease, \protect\hyperlink{trimodality}{trimodality therapy} is optimal)
\end{itemize}

Patients with dysphagia to solids or weight loss or tumor length \textgreater3cm are unlikely to have T1-2 tumors and can be evaluated with \protect\hyperlink{pet}{PET scan}.

\begin{itemize}
\tightlist
\item
  If PET shows disease confined to the esophagus and regional nodes, \protect\hyperlink{trimodality}{trimodality therapy} (chemoradiation followed by surgery) is optimal.
\item
  If PET shows metastatic disease, patients are eligible for palliative chemotherapy with radiation for treatment of symptoms of dysphagia.
\item
  If PET shows extra-regional lymph node disease, patient is at high risk for distant disease and can be treated with induction chemotherapy followed by chemoradiation and surgical evaluation.
\end{itemize}

\hypertarget{staging}{%
\chapter{Staging}\label{staging}}

The staging workup begins once a diagnosis is made on endoscopy.

The first step is to make a preliminary determination whether the tumor is early stage (and can be treated with endoscopy or primary surgery) or later stage (and treated with chemoradiation followed by surgery or with)

The diagnostic studies needed for these treatment groups are different, so the workup can be make more efficient by sorting patients at presentation in to two groups:

Patients with minimal dysphagia, no weight loss, and tumors with less than 3cm cranio-caudal extent have a reasonable change of being T1 or T2 tumors. Tumors \textless3cm in length are much more likely to represent T1-2 lesions than those \(\geq\) 3cm\citep{hollis1114}

Superficial and Localized tumors generally present with minimal dysphagia or weight loss. These tumors may present with bleeding, or dysphagia without weight loss. For these patients, determining the precise T stage is important in their workup, so \textbf{endoscopic ultrasound} is the most frequent staging study after diagnosis.

Locally-advanced or metastatic tumors tend to present with dysphagia and weight loss. At first approximation, these tumors are usually clinical T3 lesions, and the important bifurcation in their treatment is the presence or absence of metastatic disease. For patients with dysphagia and weight loss, \textbf{PET} is the most frequent initial staging study after diagnosis.

Patients who present with dysphagia are likely to have T3 or T4 disease, which is generally treated with neoadjuvant chemoradiation followed by surgery. Data from Memorial Sloan Kettering {[}Ripley 226{]} among 61 patients with esophageal cancer who presented with dysphagia, 54 (89\%) were found on EUS to have uT3-4 tumors. On the other hand, among 53 patients without dysphagia, 25 (47\%) were uT1-2, and were potentially candidates for primary surgery. Their conclusion was that EUS could be omitted from the workup of patients with dysphagia, but is useful in patients without dysphagia.

PET can be helpful in evaluating patients who may have T1-2 disease, and might be candidates for primary surgical therapy. A comparison of PET and EUS {[}malik,claxton,1{]} showed that uT1-2 tumors had median metabolic tumor volume (MTV) of 6.7cm\textsuperscript{3}, compared with uT3-4 tumors, with a median SUV of 35.7cm\textsuperscript{3}.

\hypertarget{superficial}{%
\chapter{Superficial EsoCa}\label{superficial}}

Superficial esophageal cancer is usually asymptomatic, which means that the diagnosis is generally made in the context of surveillance for Barrett's esophagus.

Nodular Barrett's esophagus can be best evaluation with endoscopic mucosal resection, which can provide further staging information if an adenocarcinoma is found, such as depth of invasion, differentiation, and lymphovascular invasion.

Larger lesions should first be evaluated with endoscopic ultrasound (EUS)?

EUS is less sensitive for T1 lesions \citep{bergeron765} -\textgreater{} use EMR for diagnosis \citep{maish1777}

(Should nodular Barrett's be evaluated with EUS prior to EMR?)

T1a tumors have a low risk of nodal metastasis \citep{dunbar850}

\hypertarget{emr}{%
\section{Endscopic Mucosal Resection (EMR)}\label{emr}}

For patients with nodular Barrett's esophagus or small tumors judged to be T1 by endoscopic ultrasound, endoscopic mucosal resection (EMR) can be diagnostic and potentially curative.\citep{thomas1609}

EMR also helps establish the difference between T1a and T1b compared with pathology \citep{worrell484}

EMR is likely sufficient for small tumors with favorable patholgic factors\citep{pech652} \citep{nurkin1090}:

\begin{itemize}
\tightlist
\item
  Size less than 2cm
\item
  Lateral and deep margins clear
\item
  Absence of lymphovascular invasion
\item
  Well- or moderately- differentiated
\end{itemize}

EMR: \citep{soetikno4490}

See MOlina JTCVS 153:1206

EMR for high-grade dysplasia \citep{shaheen2277}

EMR for low-grade dysplasia \citep{phoa1209} resulted in 25\% riskd reduction in progression go HGD.

Endoscopic submucosal dissection is a technique for deeper endoscopic removal of esophageal lesions using endoscopic cautery, which dissects through the submucosa. ESD has a higher rate of curative resection \citep{cao751} albiet at the cost of prolonged operative times and increased risk of complications such a bleeding. \citep{repici715}

ESD takes more time and has higher R0 resection rate but similar recurrence erate at 2 eyars \citep{terheggen783}

Need for RFA of Barrett's after EMR: \citep{haidry87} Combination therapy with EMR and RFA results in lower rate of recurence than EMR alone.\citep{pech1200}

RFA for Barrett's national registry \citep{ganz35}

\hypertarget{localized}{%
\chapter{Localized EsoCa}\label{localized}}

\hypertarget{t1b-tumors}{%
\section{T1b Tumors}\label{t1b-tumors}}

\hypertarget{t2n0-tumors}{%
\section{T2N0 Tumors}\label{t2n0-tumors}}

Multiple studies have failed to show the additional benefit of chemotherapy or chemoradiation for pT2N0M0 esophageal cancer patients treated with radiation.

Neoadjuvant chemo not likey to be helpful for early stage disease - FFCD 9901 {[}Mariette 2416{]} enrolled patients with T1-2 or T3N0 tumors to chemoradiation followed by surgery versus surgery alone. The majority of the tumors (72\%) were squamous cell carcinoma.Postoperative mortality was significaly increased in the chemoradiation arm (11.1\% vs 3.4\%).

Meta-analysis of 5265 patients in 10 studies showed that while neoadjuvant therapy was associated with a reduction in positive margin rate, there was no difference in terms of recurrence or survival.{[}MOta 176{]}

French trial FREGAT\citep{markar59}

Retrospective review of the National Cancer DataBase failed to demonstrate a difference in survival of cT2N0M0 esophageal cancer with or without preoperative chemoradiation.\citep{speicher1195}

A retrospective report from Johns Hopkins examined outcomes of T2N0 squamous cell carcinoma patients and showed equivalent outcomes for primary surgery vs neoadjuvant chemoradiation followed by surgery \citep{zhang429}

\hypertarget{staging-of-t2n0-tumors}{%
\section{Staging of T2N0 Tumors}\label{staging-of-t2n0-tumors}}

The challenge for treatment decision-making is the limited sensitivity of endoscopic ultrasound in ruling out pT3 or pN+ disease. In other words, if a patient who is thought to have cT2N0 disease undergoes resection, and is found on pathology to have pT3 or pN\textsuperscript{+} disease, this would dictate the need for postoperative chemoradiation. In general, chemoradiation after esophagectomy is difficult for patients to tolerate, with a \_\_\_ \% chance of failure to complete therapy.

Data from the Cleveland Clinic looked at 53 patients judged to be T2N0 by endoscopic ultrasound (uT2N0) were treated with primary surgery. Pathologic examination showed that 17 (37\%) were understaged by endoscopic ultrasound, and were pathologic (pT3) in 4 or node positive (pN\textsuperscript{+}) in 13 cases. These patients were treated with postoperative adjuvant chemoradiation.\citep{rice317}

It is critical, therefore, in patients for whom primary surgery is contemplated, to attempt to identify those with occult T3 or N+ disease.

Patients who appear to have limited stage disease benefit from evaluation with a combination of

See also \url{PMID:25047477}

(MTV)

(Tumor Length)

(dysphagia)

\#\#Primary Surgery \{\#primary\_surgery\}

NCCN recommends PET scanS

Most common sites of metastasis are liver, lung, bones, adrenal.

PET detects occult metastasis in 10-20\% of cases \citep[\citet{kim403}]{kato921}. Among 129 patients with esophageal cancer, PET detected additional sites of disease in 41\% and changed management in 38\% \citep{chatterton354}

PET for restaging detects interval development of metastatic disease in 8-17\% of cases \citep{vanvliet547}

\hypertarget{locally_advanced}{%
\chapter{Locally Advanced EsoCa}\label{locally_advanced}}

Tumors that are T2N\textsuperscript{+}M0 or T3NxM0 are considered locally-advanced. The high rate of failure with surgery alone has led to development of adjunctive therapies.

\hypertarget{trimodality}{%
\section{Trimodality Therapy}\label{trimodality}}

Trimodality therapy consists of chemoradiation followed by surgery.

CROSS trial randomized 364 patients with resectable esophageal and gastroesophageal junction tumors (75\% adenocarcinoma) to neoadjuvant chemoradiation consisting of 4,140 cGy of radiation with concurrent carboplatin and paclitaxel or surgery alone.\citep{vanhagen2074} Clinical node-positive disease was present in 16\%. Pathologic complete response was seen in 23\% of adenocarcinoma and 49\% of squamous cell carcinomas. Median overall survival was 49 months after trimodality vs 24 months after surgery alone (p=0.003). Squamous cell carcinomas appeared to have particular benefit, with a hazard ratio of 0.42 for squamous cell vs 0.74 for adenocarcinoma. Median survival was improved for adenocarcinoma from 27.1 months to 43.2 months, but the median survival for squamous cell increased from 27.1months to 81.6 months for squamous cell. Rate of R0 resection was higher with chemoradiation (92\% vs 69\% p\textless0.001) andlocal recurrence rates lower (14\% vs 34\% P\textless0.001), and peritoneal recurrence lower (4\% vs 14\% P\textless0.001). Despite the relatively low dose of radiation, in-field recurrences were less than 5\%. The primary cause of failure was distant disease (31\%) and local/regional failure (14\%).\citep{oppedijk385}

Alternative to carbotaxol is FOLFOX (SOG trial \citep{leichman4555})

Ongoing PROTECT trial compares FOLFOX to paclitaxel and carboplatin \citep{messager318}

See also \protect\hyperlink{eso_dcrt}{Definitive ChemoRT}

\hypertarget{neoadjuvant-chemort-for-scca}{%
\subsection{Neoadjuvant chemoRT for SCCA}\label{neoadjuvant-chemort-for-scca}}

NeoCRTEC5010 \citep{yang2796}

Meta-abalysis of chemoRT vs chemo \citep{zhaoe0202185}

\hypertarget{neoadjuvant-chemotheraphy-followed-by-surgery}{%
\subsection{Neoadjuvant chemotheraphy followed by surgery}\label{neoadjuvant-chemotheraphy-followed-by-surgery}}

POET Trial (Pre-Operative therapy in Esophageal adenocarcinoma Trial) treated patients with adenocarcinoma of the gastroesophageal junction with either neoadjuvant chemotherapy (5-FU, leucovorin, cisplatin) followed by surgery or induction chemotherapy with the same agents, followed by chemoradiation (4000cGy with concurrent cisplatin and etoposide). The study failed to meet its accrual goal, but there was a suggestion of improved 3-year survival with preoperative chemoradiation (47.4\% vs 27.7\% \emph{p}=0.07) as well as improved local control (76.5\% vs 59\%). In addition, chemoradiation was associated with a higher pathologic complete response rate (15.6\% vs 2\%)\citep{stahl851}. A meta-analysis of 33 randomized trials further suggested a greater benefit from neoadjuvant chemoradiation followed by surgery compared with neoadjuvant chemotherapy followed by surgery\citep{pasquali481} and a similar meta-analysis \citep{sjoquist681}

\#Active Surveillance

EGD is poor predictor of pCR \citep{sarkaria764}

\hypertarget{ge-junction}{%
\section{GE Junction}\label{ge-junction}}

\citep{siewert260}

\hypertarget{induction-chemotherapy-followed-by-chemort}{%
\section{Induction chemotherapy followed by chemoRT}\label{induction-chemotherapy-followed-by-chemort}}

See NCCN pages M-25 and M-26

Stahl \citep{stahl851} randomized patients to preoperative chemotherapy (A) vs preoperative chemotherapy followed by preoperative chemoradiation (B). Higher pcR rate in arm B (15.6\% vs 2\%) and ypN0 resection (64.4\% vs 37.7\%).

\hypertarget{postoperative-chemoradiation}{%
\section{Postoperative chemoradiation}\label{postoperative-chemoradiation}}

Intergroup-0116 \citep{macdonald725} \citep{smalley2327} treated 556 patients with adenocarcinoma of the stomach or GE junction with surgery along vs surgery followed by postoperative chemoradiation. After a median followup of over 5 years, median overall survival iin the surgery alone group was 27 months vs 36 months in the postoperative chemoradiation group (p=0.005) Decrease in local failure as the first site of failure in the chemoradiation group (19\% versus 29\%).

Chemoradiation afte resectdion of GE junction tumors \citep{kofoed26} among a group of 211 patients with GE junction adenocarcinoma with positive lymph nodes with improved 3-year disease-free survival (37\% s 24\%).

\hypertarget{eso_dcrt}{%
\chapter{Chemoradiation}\label{eso_dcrt}}

**This section addresses chemoradiation as primary (definitive) therapy for esophageal cancer. See also \protect\hyperlink{trimodality}{Trimodality Therapy}

\hypertarget{phase-ii-studies}{%
\section{Phase II Studies}\label{phase-ii-studies}}

Experience with patients who refuse surgery or are medically unfit:

MD Anderson report of 61 patients out of 622 tri-modality-eligible patients who refused surgery after cCR. 5-year overall survival was 58\%. 13 developped local recurrence during surveillance, and 12 had successful salvage esophagectomy \citep{taketa300}. The same group compared those who underwent surgery vs definitive chemoradiation in a propensity-matched fashion\citep{taketa95}. Irish study of 56 patients aged 70 or older treated with neoadjuvant chemoradiation followed by selective surgery. Median survival was 28 months overall, 47 months for those with cCR, 61 months for primary resection, 46 months for cCRs who did not undergo resection, and 29 months for those with salvage esophagectomy\citep{furlong107}.

Castoro\citep{castoro1375}

preSANO\citep{chirieac1347} Clinical Response evaluation after chemoRT for esophageal cancer with PET and EGD.

\hypertarget{chemort-vs-trimodality-therapy}{%
\section{ChemoRT vs Trimodality therapy}\label{chemort-vs-trimodality-therapy}}

The sensitivity of squamous cell carcinoma of the esophagus to chemoradiation has raised the question whether

Stahl Locally advanced squamous cell carcinoma randomized to induction chemotherapy (cisplatin, etopiside, 5FU with leuocovrin) followed by chemoradiation (4000cGy with concurrent ciplatin and etopiside) followed by surgery compared with induction chemotherapy followed by chemoradiation (6400cGy with concurrent cisplatin and etopiside).\citep{stahl2310} progression-free survival was better in the trimodality group (64.3\% vs 40.7\%) Treatment-related morality was substantial in the surgery arm (13\% vs 4\%). This would be considered an excessive rate of operative mortality by modern standards. Unsurprisingly, there was no difference in overall survival between groups, in part because the surgical group had an excess 9\% mortality rate from treatment. Two-year survival in the surgery arm was 40\% vs 35\% in the definitive chemoradiation arm.

In the French FFCD trial, 444 patients with carcinoma of the esophagus (90\% squamous cell) were treated with two cycles of 5-FU and cisplatin with concurrent radiation.\citep{bedenne1160} Patients with a partial or complete clinical response to chemoradiation were randomized to either surgery or a boost of radiation. Patients who did not respond to chemoradiation were treated with surgery and were eliminated from the study. Only 259 of the original 444 patients (59\%) went on to randomization, with the remainder (those not responding to chemoradiation) treated with surgery. Of the randomized group, median survival was 17.7months in the surgery arm versus 19.3months in the definitive chemoradiation arm. Like the Stahl study, treatment-related mortality in the surgical arm was high (9\% versus 1\%).

\#Active Surveillance

EGD is poor predictor of pCR \citep{sarkaria764}

\hypertarget{radiation}{%
\chapter{Radiation}\label{radiation}}

RTOG 94-05 clinical trial \citep{minsky1167}

\hypertarget{esophagectomy}{%
\chapter{Esophagectomy}\label{esophagectomy}}

Three general approaches exist for surgical therapy.

Trans-thoracic or Ivor Lewis esophagectomy\citep{visbal1803} removes the intrathoracic portion of the esophagus and constructs an anastomosis within the chest. The approach include an abdominal phase, during which an esophageal substitute is constructed (usually from stomach). A thoracic phase then removes the intrathoracic esophagus and constructs an anastomosis within the chest cavity.

A McKeown esophagectomy utilizes three surgical fields: abdomen, right chest, and neck. The right chest approach allows dissection of peri-esophageal lymph nodes, and the cervical incision allows removal of the total esophagus.\citep{mckeown259} This approach is useful for tumors which involve the proximal thoracic esophagus, to ensure a negative margin. The cervical anastomosis carries a higher risk of anastomotic leak than a thoracic anastomosis, although the morbidity of a cervical anastomosis leak is less serious than that of a leak of a thoracic anastomosis.

A transhiatal esohpagectomy approaches the esophagus from the abdomen through the hiatus and from neck. By blunt dissection the esophagus is freed up without the need for thoracotomy. An esophageal substute is then brought from the abdomen to the neck through the mediastinum\citep{orringer643} \citep{orringer363}\textless! -- Orringer Ann Surg 2007 --\textgreater{} The operation is designed to avoid the pulmonary toxicity of the right chest approach. On the other hand, the blunt nature of the mediastinal dissection means that fewer lymph nodes are harvested than with a trans-thoracic approach.

Randomized trial of transthoracic esophagectomy with extended lymph node dissection versus transhiatal esohpagectomy showed fewer pulmonary complications with the transhiatal approach. \citep{hulscher1662} Fewer lymph nodes were havested with a transhiatal appraoch. A post-hoc analysis showed that among patients with 1-8 positive lymph nodes, survival with improved with the extended lymph node dissection.\citep{omloo992}

Minimally-invasive approaches to esophagectomy are now common, with evidence for less perioperative morbidity than an open approach \citep{biere1887} \citep{zhoue0132889}

Randomized trial of a hybrid MIE (with laparoscopy and thoracotomy) was associated with lower postoperative complications than open esophagectomy \citep{mariette152}

High volume \citep{birkmeyer2117} \citep{wouters1789}

Siwert III lesions are considered gastric cancers \citep{rusch444} \citep{siewert260}

Laparoscopy may be helpful in Siewert III tumors \citep{degraaf988}

\hypertarget{preoperative-evaluation}{%
\subsection{Preoperative Evaluation}\label{preoperative-evaluation}}

Dysphagia can be scored according to Mellow et al \citep{mellow1443}:

\begin{itemize}
\tightlist
\item
  0 No dysphagia
\item
  1 Dysphagia to normal solids
\item
  2 Dysphagia to soft solids (ground beef, poultry,fish)
\item
  3 Dysphagia to solids and liquids
\item
  4 Inability to swallow saliva
\end{itemize}

\hypertarget{minimally-invasive-esophagectomy}{%
\section{Minimally-invasive Esophagectomy}\label{minimally-invasive-esophagectomy}}

Higher lymph node yield with MIE vs open approach \citep{kalffa}

\hypertarget{transthoracic}{%
\section{Transthoracic}\label{transthoracic}}

\hypertarget{transhiatal}{%
\section{Transhiatal}\label{transhiatal}}

\hypertarget{three-hole}{%
\section{Three-hole}\label{three-hole}}

\hypertarget{extended-lymphadenectomy}{%
\section{Extended lymphadenectomy}\label{extended-lymphadenectomy}}

\hypertarget{early-recovery-pathways}{%
\section{Early Recovery Pathways}\label{early-recovery-pathways}}

\href{https://www.ncbi.nlm.nih.gov/pmc/articles/PMC7098920/}{ERAS Society Guidlines}

\hypertarget{salvage-esophagectomy}{%
\section{Salvage esophagectomy}\label{salvage-esophagectomy}}

\citep{markar922}

\citep{swisher175}

\hypertarget{eso_metastatic}{%
\chapter{Metastatic EsoCa}\label{eso_metastatic}}

\hypertarget{palliative-radiation}{%
\section{Palliative radiation}\label{palliative-radiation}}

Palliative radiation vs chemoradiation \citep{penniment114}

Radiation along favored over chemoradiation in the palliaitve setting \citep{penniment114}

\hypertarget{chemoradiation-vs-chemotherapy-in-stage-iv}{%
\section{Chemoradiation vs chemotherapy in Stage IV}\label{chemoradiation-vs-chemotherapy-in-stage-iv}}

\citep{guttmann1131}

\hypertarget{stents-for-malignant-disease}{%
\section{Stents for malignant disease}\label{stents-for-malignant-disease}}

\citep{vakil1791}

Review of guidelines 2010 Am Society GI \citep{sharma258}

\hypertarget{eso_survivorship}{%
\chapter{Survivorship}\label{eso_survivorship}}

\hypertarget{nutritional-consequences-of-esophagectomy}{%
\section{Nutritional consequences of esophagectomy}\label{nutritional-consequences-of-esophagectomy}}

\hypertarget{vitamin-d-deficiency}{%
\subsection{Vitamin D deficiency}\label{vitamin-d-deficiency}}

Vitamin D deficiency is defined as serum 25(OH)D levels below 20ng/mL

\begin{itemize}
\tightlist
\item
  Replacement with 2000 IU Vitamin D3 daily \citep{khan97}
\end{itemize}

Vitamin D insufficiency is defined as serum 25(OH)D level 20-29ng/mL

\begin{itemize}
\tightlist
\item
  Replacement with 1000-2000 IU Vitamin D3 daily
\end{itemize}

\href{https://ods.od.nih.gov/factsheets/vitamind-healthprofessional/}{NIH ODS Vitamin D info}

\citep{baker987}

Weight loss
\citep{martin1308}
\citep{ouattara1088}

\hypertarget{cardiac-toxicity-of-radiation}{%
\section{Cardiac toxicity of radiation}\label{cardiac-toxicity-of-radiation}}

\citep{beukema85} \citep{frandsen516} \citep{gharzaie0158916}

\hypertarget{eso_surveillance}{%
\section{Surveillance}\label{eso_surveillance}}

\textbf{T1a treated with endoscopic resection}
EGD every 3 mo for first year, then every 6 months for second year, then annually\citep{shaheen30}

\textbf{Tib treated with endoscoic resection}
EGD every 3 mon for first year, then every 4-6 months for seond year, then annually
CT chest/abdomen every 12 months for up to 3 years (as clinically indicated)

\textbf{T1b treated with esophagectomy}
EGD every 3-6 months for first 2 years, then annually for 3 more years.
CT every 6-9 months for first 2 years, then annually up to 5 years.

\textbf{Stage II or III treated with chemoradiation.}

These patients are at risk for local recurrence \citep{sudo3400} and some may be candidates for salvage esophagectomy. Most relapses (95\%) occur within 24 months. See also \citep{taketa1139}

\textbf{Locally-advanced treated with trimodality therapy}

Local/regional relapses are uncommon. \citep{dorth2099} \citep{oppedijk385} \citep{sudo4306} =\textgreater{} NCCN does not recommend EGD. 90\% of relapses occur within 36 months of surgery.

CT every 6 months up to 2 years (if patient is a candidate for additional curative-intent therapy)

\hypertarget{part-gastric-cancer}{%
\part*{Gastric Cancer}\label{part-gastric-cancer}}
\addcontentsline{toc}{part}{Gastric Cancer}

\hypertarget{gastric-overview}{%
\chapter{Gastric Overview}\label{gastric-overview}}

\hypertarget{gast_superficial}{%
\chapter{Superficial Gastric}\label{gast_superficial}}

\hypertarget{locally-advanced-gastric}{%
\chapter{Locally-Advanced Gastric}\label{locally-advanced-gastric}}

Locally-advanced gastric cancer (T3 or N\textsuperscript{+}) is generally treated with some form of adjuvant therapy, which has been shown to improve upon the outcomes with surgery alone.

\hypertarget{preoperative-chemotherapy}{%
\section{Preoperative Chemotherapy}\label{preoperative-chemotherapy}}

FLOT chemotherapy \citep{al-batran1948}

MAGIC study randomized 503 patients to perioperative `sandwich' therapy consisting of epirubicin, cisplatin, and 5-FU versus surgery alone. In the perioperative chemotherapy group, 4 cycles were administered prior to surgery, and 4 cycles afterwards. Tumors of the esophagus or gastroesophageal junction comprised 26\% of the study population. While over 90\% of patients assigned to the chemotherapy arm completed their preoperative chemotherapy, only 66\% completed their postoperative therapy. Survival at 5 years was 36\% in the perioperative chemotherapy group, compared with 24\% in the surgery group (p\textless0.001).\citep{cunningham11}

CLASSIC clinical trial randomized 1033 patients with stage II or III gastric cancer after D2 gastrectomy to 6 months of adjuvant chemotherapy versus surgery alone. Three-year survival was improved in the chemotherapy group (74\% \emph{v} 59\%).\citep{bang315}

The FFCD trial randomized patients to preoperative chemotherapy with 2 or 3 yccles of cisplatin and 5-FU versus surgery alone. Tumors of the lower esophagus or gastroesophageal junction comprised 75\% of the study population. Survival at 5 years was longer in the chemotherapy group (38\%) versus 24\% in the surgery alone group (p=0.02).\citep{ychou1715}

\hypertarget{postoperative-chemotherapy}{%
\section{Postoperative chemotherapy}\label{postoperative-chemotherapy}}

CLAASIC trial \citep{noh1389} \citep{bang315} patients with II or IIIB gastric cancer received gastrectomy with D2 node dissection randomized to postoperative chemotherapy with capecitabine and oxaliplatin. Chemotherapy group had improved 3-year DFS (74\% vs 59\% P\textless.0001)

\hypertarget{postoperative-chemoradiation-1}{%
\section{Postoperative chemoradiation}\label{postoperative-chemoradiation-1}}

Intergroup 0116 trial \citep{macdonald725}
Surgical quality control was poor, as 90\% were treated a limited lymph node dissection. Long-term followup, however \citep{smalley2327} showed a persistent benefit of postoperative chemoradiation.

ARTIST trial 450 patients treated with a D1 \(\alpha\) gastrectomy were randomized to adjuvant capcitibine and cisplatin versus chemoradiation consisting of two cycles of capcitabine/oxalipaltin followed by chemoradiation followed by chemotherapy. Overall 3- year survival did differ between groups (78.2\% vs 74.2\% p =0.86). A post-hoc analysis of patients with positive nodes showed a beneficial effect of chemoradiation (77.5\% \emph{v} 72.3\% p=0.365).\citep{lee268}

CRITICS trial treated all patients with preoperative chemoterhapy followed by surgery. Postoperative patients were then randomized between additional chemotherapy versus chemoradiation.

\hypertarget{preoperative-chemoradiation}{%
\section{Preoperative chemoradiation}\label{preoperative-chemoradiation}}

\citep{ajani3953}

\hypertarget{hereditary-diffuse-gastric-cancer}{%
\chapter{Hereditary Diffuse Gastric Cancer}\label{hereditary-diffuse-gastric-cancer}}

\hypertarget{part-colon-cancer}{%
\part*{Colon Cancer}\label{part-colon-cancer}}
\addcontentsline{toc}{part}{Colon Cancer}

\hypertarget{colorectal-overview}{%
\chapter{Colorectal Overview}\label{colorectal-overview}}

\textbf{Resources}

\href{https://fascrs.org/ascrs/media/files/downloads/2022-Colon-Cancer-CPG.pdf}{ASCRS Guidelines for treatment of Colorectal Cancer 2022}

\href{https://fascrs.org/ascrs/media/files/downloads/Clinical\%20Practice\%20Guidelines/clinical_practice_guidelines_for_enhanced_recovery-3.pdf}{ASCRS Guidelines for Early Recovery}

\hypertarget{colon-cancer-staging}{%
\chapter{Colon Cancer Staging}\label{colon-cancer-staging}}

\hypertarget{cea-level}{%
\section{CEA level}\label{cea-level}}

CEA should be routinely checked prior to surgery

\hypertarget{ct-scan}{%
\section{CT Scan}\label{ct-scan}}

\hypertarget{pet-scan}{%
\section{PET scan}\label{pet-scan}}

PET is not recommended routinely for evaluation of colon cancer.\citep{vogel148}

\hypertarget{indications-for-pet-in-colorectal-cancer}{%
\subsection{Indications for PET in colorectal cancer:}\label{indications-for-pet-in-colorectal-cancer}}

\begin{itemize}
\tightlist
\item
  Workup of rising CEA
\item
  Indeterminate extra-hepatic findings on CT or MRI
\item
  Evaluation of possible local recurrence
\end{itemize}

\hypertarget{adjuvant-therapy}{%
\section{Adjuvant Therapy}\label{adjuvant-therapy}}

Adjuvant therapy for stage II colon cancer Review \{28399381\}

\hypertarget{stage-iv-colon-cancer}{%
\chapter{Stage IV Colon Cancer}\label{stage-iv-colon-cancer}}

\hypertarget{carcinomatosis}{%
\section{Carcinomatosis}\label{carcinomatosis}}

\hypertarget{colon-resection-in-stage-iv-colon-cancer}{%
\section{Colon resection in stage IV colon cancer}\label{colon-resection-in-stage-iv-colon-cancer}}

MSKCC: Colon surgery patients with bilobar liver metastasis treated with chemotherapy with primary tumor in place. Surgical intervention needed in 7\% of cases \citep{poultsides35}

\hypertarget{immunotherapy-for-dmmr-msi-high}{%
\section{Immunotherapy for dMMR (MSI high)}\label{immunotherapy-for-dmmr-msi-high}}

Colorectal cancers deficient in mismatch-repair proteins (dMMR) respond poorly to 5FU-based chemotherapy but do respond to immune-checkpoint inhibitors (\(\alpha\)-PD-1, (\(\alpha\)PD-L1, (\(\alpha\)-CTLA-4).

\href{https://www.frontiersin.org/articles/10.3389/fimmu.2022.795972/full}{Review} of immune checkpoint inhibitors in colorectal cancer

Blockade of the PD-1 (Programmed Death-1) ligand with monoclonal antibody as shows efficacy in colorectal cancer, especially those deficient in mismatch repair protein expression, as manifested by microsatellite instability (MSI-High) on pathologic exam.

Keynote-177 trial \citep{andre2207} randomized 307 patients with MSI-High dMMR colorectal cancer patients to receive either pembrolizumab or 5-FU based chemeotherapy and found superior survival with pembro.

\hypertarget{colectomy-1}{%
\chapter{Colectomy}\label{colectomy-1}}

\hypertarget{extended-node-dissection}{%
\section{Extended Node dissection}\label{extended-node-dissection}}

Short-term outcomes of complete mesocolic excision versus D2 dissection in patients undergoing laparoscopic colectomy for right colon cancer (RELARC): a randomised, controlled, phase 3, superiority tria

Short-term outcomes of a multicentre randomized clinical trial comparing D2 versus D3 lymph node dissection for colonic cancer (COLD trial).
Karachun A, Panaiotti L, Chernikovskiy I, Achkasov S, Gevorkyan Y, Savanovich N, Sharygin G, Markushin L, Sushkov O, Aleshin D, Shakhmatov D, Nazarov I, Muratov I, Maynovskaya O, Olkina A, Lankov T, Ovchinnikova T, Kharagezov D, Kaymakchi D, Milakin A, Petrov A.
Br J Surg. 2020 Apr;107(5):499-508. doi: 10.1002/bjs.11387. Epub 2019 Dec 24.
PMID: 31872869 Clinical Trial.

\hypertarget{chemotherapy}{%
\chapter{Chemotherapy}\label{chemotherapy}}

\hypertarget{neoadjuvant-chemotherapy}{%
\section{Neoadjuvant Chemotherapy}\label{neoadjuvant-chemotherapy}}

Seymour MT, Morton D. FOxTROT: an international randomized controlled trial in 1052 patients (pts) evaluating neoadjuvant chemotherapy (NAC) for colon cancer. J Clin Oncol. 2019 May;37(15 Suppl):3504-3504.

\hypertarget{part-rectal-cancer}{%
\part*{Rectal Cancer}\label{part-rectal-cancer}}
\addcontentsline{toc}{part}{Rectal Cancer}

\hypertarget{rectal-overview}{%
\chapter{Rectal Overview}\label{rectal-overview}}

\textbf{Resources}

\href{https://fascrs.org/ascrs/media/files/downloads/2022-Colon-Cancer-CPG.pdf}{ASCRS Guidelines for treatment of Colorectal Cancer 2022}

\hypertarget{rectal-cancer-staging}{%
\chapter{Rectal Cancer Staging}\label{rectal-cancer-staging}}

\hypertarget{mri-staging}{%
\section{MRI staging}\label{mri-staging}}

MRI has supplanted endoscopic ultrasound (EUS) as the study of choice for staging rectal cancer.

MERCURY group trial demonstrated the predictive value of MRI for rectal cancer\citep{taylor34} \citep{mercurystudygroup779}

\hypertarget{rectal-cancer-surgery}{%
\chapter{Rectal Cancer Surgery}\label{rectal-cancer-surgery}}

Importance of total mesorectal excision was championed by Bill Heald \citep{heald1479} who emphasized sharp dissection of the mesorecum outside of the visceral fasial envelope. In addition he advocated dissection of the totality of the mesorectum distal to the tumor to avoid leaving behind nodes within the mesorectum distal to the tumor\citep{quirke996} \citep{nagtegaal303} \citep{paty365}.

\hypertarget{rectal-adjuvant-therapy}{%
\chapter{Rectal Adjuvant Therapy}\label{rectal-adjuvant-therapy}}

German Rectal Cancer Study (AIO-94) compared preop and postop chemoradiation in locally-advanced rectal cancer \citep{sauer173}. Among 823 patients, local recurrence was 6\% in the preoperative group vs 13\% in the postop group (p=0.0006). Overall survival was not different, but preop chemoradiation was less toxic. Pathologic complete response rate in the preoperative group as 8\%.

NSABP R-03 clinical trial \citep{roh5124} randomized 267 patients with T3 or T4 or node-positive rectal cancer to preop vs postop chemoradiation. Disease-free survival was better in the preoperative group, wiht no difference in overall survival. Pathologic complete response rate in the preop group was 15\%.

Dutch Rectal Cancer Trial (CKVO 95-04): Demostrated the benefit of preop radiation in combination with TME\citep{kapiteijn638}. Radiation reduced local recurrence form 10.9\% to 5.6\%

\hypertarget{short-course-preoperative-therapy}{%
\subsection{Short-course preoperative therapy}\label{short-course-preoperative-therapy}}

Swedish Rectal Cancer Trial compared surgery alone with preoperative short-course therapy consisting of 500cGy of radiation without chemotherapy administered in one week prior to surgery. Local recurrence was 9\% in the therapy group vs 26\% in the control group, with an improvement in overall survival of 38\% vs 80\%\citep{swedishrectalcancertrial980}. Of note, this trial was performed in the era prioro to the widespread use of total mesorectal excision.

Stockholm III trial showed that short-course radiation therapy performed 4-8 weeks prior to surgery resulted in imporved rates of pathologic complete response (12\% vs 2\%) compared with short-course radiation therapy performed the week prior to surgery \citep{pettersson972}

The Dutch trial found a reduction in local recurrence form 10.9\% to 5.6\% with preoperative radiation with no difference in survival\citep{kapiteijn638}. A later update\citep{vangijn575} The Medical Research Council C07 trial compared preoperative short-course radiation with selective post-operative chemoradiation and found no difference in local recurrence (4.4\% vs 10.6\%) and no difference in overall survival.\citep{sebag-montefiore811}

\hypertarget{total-neoadjuvant-therapy}{%
\subsection{Total Neoadjuvant therapy}\label{total-neoadjuvant-therapy}}

RAPIDO trial randomized 920 patients with T4 or node-positive disease to long-course chemoradiation followed by surgery vs short-course radiation followed by chemotherapy and surgery. The pCR rate was significantly higher in the short course/chemotherapy/surgery group (28\% vs 14\%) and disease-specific surival at 3years was higher (30\% vs 24\%).\citep{bahadoer29} \citep{vandervalk75}

PRODIGE 23 randomized 461 patients with T3 or T4 recta cancers to long-course radiation followed by surgery vs induction chemotherapy, long-course radiation followed by surgery. Up-front chemotherapy was associates with increased 3-year survival (76\% vs 69\%) and an increase in rate of pathologic complete response of 28\% vs 12\%.\citep{conroy702}

STELLAR trial \citep{jin1681} TNT with short-course radiation (5Gy x 5 days) + CAPEOX chemotherapy vs long-course chemoRT with capecitibine. No differences in relapse-free survival or metastasis free survival. Overall survival better with TNT (p=0.03). On subgroup analysis, no group appeared to have more benefit. pCR rate 17\% with TNT vs 13.9\% with conventional chemoradiation. Positive margins (R1) occurred in 8.5\% of TNT patients vs 12.5\% in the conventional chemoRT group.

OPRA clinical trial \citep{garcia-aguilarJCO2200032}. 324 rectal cancer patients staged with MRI randomized to INCT (induction chemo followed by chemoRT) vs CNCT (Chemoradiation followed by chemotherapy). Patients with a resopnse were offered watch and wait. Patients without a response were treated with surgery. No difference in disease-free survival or overall survival. Among patients in watch and wait, somewhat more patients with INCT had recurrences (42 patients vs 32 patients). More organ preservation with CNCT. See also \citep{smith767}

\hypertarget{neoadjuvant-immunotherapy-msi-high}{%
\section{Neoadjuvant immunotherapy (MSI high)}\label{neoadjuvant-immunotherapy-msi-high}}

MSKCC trial of neoadjuvant PD-1 blockade with 6 months of dostarlimab for 12 patients with MMR-deficient (MSI-high) rectal cancer resulted in 100\% clinical response rate \citep{cercek2363}. No patients were subsequently treated with either chemoradiation or surgery as originally planned.

PICC trial \citep{hu38} Chinese trial of PD-1 blockade with neoadjvant toripalimab + colecoxib vs toripalimab for 3 months in 34 patients with MMR-deficient locally-advanced (T3/4 or N+) colorectal cancer. All patients were then treated with surgery. Pathologic complete response in 88\% of dual-therapy patients and 65\% of monotherapy. Grade 3 toxicity in 1/34 patients. Previous neoadjuvant chemo in 25\%. Rectal cancer in 18\%.

\hypertarget{non-operative-management-of-rectal-cancer}{%
\section{Non-operative management of Rectal Cancer}\label{non-operative-management-of-rectal-cancer}}

MSKCC published an experience of 113 patients with cCR after chemoradiation for rectal cancer who elected non-operative management \citep{smithe185896}. Among this group, 22 developed a local recurrence, and 8\% developed metastatic disease.\citep{smith657}

\hypertarget{part-sarcoma}{%
\part*{Sarcoma}\label{part-sarcoma}}
\addcontentsline{toc}{part}{Sarcoma}

\hypertarget{soft-tissue-sarcomas}{%
\chapter{Soft Tissue Sarcomas}\label{soft-tissue-sarcomas}}

\hypertarget{desmoid-tumors}{%
\section{Desmoid Tumors}\label{desmoid-tumors}}

Review of desmoid tumors: \href{https://www.ejcancer.com/article/S0959-8049(19)30832-9/fulltext}{Global Consensus Guidelines}

\hypertarget{retroperitoneal}{%
\section{Retroperitoneal}\label{retroperitoneal}}

\textbf{Preop Radiation}

STRASS trial randomized 266 patients to preop radiation followed by surgical resection vs surgery alone. At a median followup of 43 months, there was no difference in recurrence-free survival. Serious adverse affects were more common in the preop radiation group (24\% vs 10\%). One patients in the radiation group died of treatment-related toxicity (gastrocolic fistula), compared with none in the surgery alone group \citep{bonvalot1366}. See commentary: \citep{cardona1257}

STRASS2: Ongoing trial of neoadjuvant chemotherapy.

  \bibliography{zotero.bib}

\end{document}
